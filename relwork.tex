Basic Linear Algebra Subprograms (BLAS) defines a small set of vector and
matrix operations commonly used in scientific computing. There exist a few
highly-optimized BLAS implementations such as MKL \cite{mkl}, OpenBLAS
\cite{openblas}, GotoBLAS \cite{Goto} and ATLAS \cite{atlas}. However, these
libraries optimize the vector and matrix operations in shared memory. BLAS
provides only a small number of vector and matrix operations.

Distributed memory matrix computation libraries \cite{trilinos, petsc, elemental}
speed up computation and scale to larger vectors and matrices. These libraries
in general build on top of BLAS and distribute computation with MPI.
They provide a limited set of predefined matrix operations and
require users to manually parallelize the remaining matrix operations. Instead
of providing predefined matrix operations, the core of FlashMatrix provides
a few GenOps that represent some common data access patterns. By accepting
different functions that define operations on individual elements in the input
matrices, each GenOp covers a very large number of matrix operations.

There are many distributed data processing frameworks.
MapReduce \cite{mapreduce} is a general large-scale data processing framework.
It provides a single primitive that takes two user-defined functions. 
% RB not true
%written in low-level programming languages such as C/C++ and Java. 
Due to the lack of
efficient primitives for varieties of data access patterns, algorithms
implemented in MapReduce are inefficient. Dryad \cite{dryad} and
Naiad \cite{naiad} provide more primitives than MapReduce to support various
data access patterns more efficiently.

Due to complexity of programming in the distributed execution engines, many
programming frameworks have been developed on top of the distributed execution
engines. Pig Latin \cite{pig} and FlumeJava \cite{flumejava} build on top of
MapReduce to provide high-level operations for general data analysis. SystemML
\cite{systemml} builds on top of MapReduce with a focus on machine learning.
DryadLINQ \cite{dryadlinq} builds on top of Dryad and exposes a high-level
language to express data analysis tasks. The performance of these programming
frameworks is bound by the underlying distributed execution engines.

Spark \cite{spark} is a distributed in-memory data processing framework.
It provides a highly-optimized machine learning library called MLlib \cite{mllib}.
Spark also provides an R programming interface called SparkR, which
focuses on computation on data frames, a table-like data structure in R.
% DZTODO  what are data frames?

Both academia and industry are making significant effort to bring parallelization to
array programming languages and scale them to large datasets. Revolution R
\cite{rre} and parallel computing toolbox in MatLab \cite{matlab} provide
parallel linear algebra and data analysis routines as well as explicit
parallel programming interface such as MPI and MapReduce. Other works bring
implicit parallelization to programming frameworks. Presto \cite{presto}
extends R to support sparse matrix operations in distributed memory for graph
analysis. Ching et. al \cite{Ching12} parallelizes APL code by
compiling it to paralllelized C code. Accelerator \cite{accelerator} compiles
data-parallel operations on the fly to execute programs in GPU.
