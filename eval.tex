\section{Experimental evaluation}
We evaluate the performance of FlashMatrix with statistics and machine learning
applications both in memory and on SSDs. We compare their performance with
the implementations in Spark MLlib \cite{mllib}, a high-optimized parallel
machine learning library. We further
illustrate the effectiveness of the optimizations deployed in FlashMatrix
when running both in memory and on SSDs.

We conduct experiments on a non-uniform memory architecture machine with
four Intel Xeon E7-4860 processors, clocked at 2.6 GHz, and 1TB memory of
DDR3-1600. Each processor has 12 cores. The machine has three LSI SAS 9300-8e
host bus adapters (HBA) connected to a SuperMicro storage chassis, in which
24 OCZ Intrepid 3000 SSDs are installed. The 24 SSDs together are capable of
delivering 12 GB/s for read and 10 GB/s for write at maximum. The machine runs
Linux kernel v3.13.0. By default, we use 48 threads for both in-memory and
out-of-core execution of FlashMatrix.

\subsection{Statistics and Machine learning applications} \label{sec:apps}
We implement multiple important applications in the field of statistics and
machine learning. These applications have various computation complexity but
have the same I/O complexity (Table \ref{tbl:algs}). We implement these
applications completely with the R interface of FlashMatrix and
rely on FlashMatrix to perform computation in parallel and out of core.
\begin{itemize}
	\item Multivariate statistical summary: this computes column-wise minimum,
		maximum, mean, L1 norm, L2 norm, the number of non-zero values and
		variance on a data matrix.
	\item Correlation: this computes pair-wise Pearson's correlation \cite{}
		among multiple series of data and is commonly used in statistics.
	\item Singular value decomposition (SVD) factorizes a matrix into
		three matrices: $U$, $\Sigma$ and $V$ such that $A=U \Sigma V^T$, where
		both $U$ and $V$ are orthonormal matrices and $\Sigma$ is a diagonal
		matrix with non-negative diagonals in descending order. SVD is commonly
		used for dimension reduction. In the experiments, we compute 10 left
		singular vectors and 10 right singular vectors.
	\item K-means \cite{kmeans} is an iterative algorithm of partitioning a set
		of data points into $k$ clusters so that each cluster has minimal mean
		of distances between the data points in the cluster and the cluster
		center. K-means
		is one of the most popular clustering algorithms and is identified as
		one of the top 10 data mining algorithms \cite{top10}. In the experiments,
		we run k-means to split a dataset into 10 clusters by default.
	\item Gaussian Mixture Model (GMM) \cite{gmm} is another iterative clustering
		algorithm that assumes data points are sampled from a mixture of
		Gaussian distributions and use expectation maximization (EM) \cite{gmm}
		algorithm to fit the model. This algorithm is also identified as one
		of the top 10 data mining algorithms \cite{top10}. In the experiments,
		we run GMM to split a dataset into 10 clusters by default.
\end{itemize}

\begin{table}
\begin{center}
\footnotesize
\begin{tabular}{|c|c|c|c|c|}
\hline
Application & Computation & I/O \\
\hline
Summary & $O(n \times p)$ & $O(n \times p)$ \\
\hline
Correlation & $O(n \times p^2)$ & $O(n \times p)$ \\
\hline
SVD & $O(n \times p^2)$ & $O(n \times p)$ \\
\hline
K-means (1 iteration) & $O(n \times p \times k)$ & $O(n \times p)$ \\
\hline
GMM (1 iteration) & $O(n \times p^2 \times k)$ & $O(n \times p + n \times k)$ \\
\hline
\end{tabular}
\normalsize
\end{center}
\caption{The computation and I/O complexity of the algorithms for the five
	applications. $n$ is the number of data points in the dataset, $p$ is
	the number of the attributes and $k$ is the number of clusters k-means and
GMM partition the dataset.}
\label{tbl:algs}
\end{table}

K-means and GMM typically run on a dataset with a small number of attributes
due to curse of dimensionality \cite{} while the other applications may
be applied to datasets with various numbers of attributes. For performance
evaluation, we run k-means and GMM on a matrix
with 64 million rows and 32 columns, constructed from 32 eigenvectors from
the Friendster graph \cite{friendster}. Running k-means on eigenvectors is
an important step of spectral clustering \cite{Luxburg07} to partition vertices
to clusters. In addition, we generate a random dataset with one billion data
points and 32 attributes, sampled from 10 mixtures of Gaussian distributions
with the same covariance matrix and different means. We measure the performance
of the other three applications on random matrices with 64 million rows
and vary the number of columns from 8 to 512.

\subsection{Comparative performance of FlashMatrix}

We compare the performance of the FlashMatrix implementations of the applications
with the ones in Spark MLlib \cite{mllib} and R. We run the MLlib implementations
with their native Scala interface and use a very large heap size to ensure that
all input data is cached in memory. We use 48 threads for both FlashMatrix and
MLlib to run on a random matrix with one billion rows and 32 columns.
The R framework provides C implementations for correlation, SVD and k-means.
The R package mclust \cite{mclust} provides a FORTRAN implementation of GMM.
These implementations run in a single thread. To compare with these implementations,
we run the FlashMatrix implementations in a single thread on a random matrix
with 64 million rows and 32 columns.

\begin{figure}
	\centering
	\footnotesize
	\vspace{-15pt}
	\begin{subfigure}{.5\textwidth}
		\begin{tikzpicture}[gnuplot]
%% generated with GNUPLOT 4.6p4 (Lua 5.1; terminal rev. 99, script rev. 100)
%% Fri 08 Apr 2016 06:45:40 PM EDT
\path (0.000,0.000) rectangle (8.382,4.572);
\gpcolor{color=gp lt color border}
\gpsetlinetype{gp lt border}
\gpsetlinewidth{1.00}
\draw[gp path] (1.688,0.616)--(1.868,0.616);
\draw[gp path] (7.829,0.616)--(7.649,0.616);
\node[gp node right] at (1.504,0.616) { 0.1};
\draw[gp path] (1.688,0.830)--(1.778,0.830);
\draw[gp path] (7.829,0.830)--(7.739,0.830);
\draw[gp path] (1.688,1.113)--(1.778,1.113);
\draw[gp path] (7.829,1.113)--(7.739,1.113);
\draw[gp path] (1.688,1.258)--(1.778,1.258);
\draw[gp path] (7.829,1.258)--(7.739,1.258);
\draw[gp path] (1.688,1.327)--(1.868,1.327);
\draw[gp path] (7.829,1.327)--(7.649,1.327);
\node[gp node right] at (1.504,1.327) { 1};
\draw[gp path] (1.688,1.542)--(1.778,1.542);
\draw[gp path] (7.829,1.542)--(7.739,1.542);
\draw[gp path] (1.688,1.825)--(1.778,1.825);
\draw[gp path] (7.829,1.825)--(7.739,1.825);
\draw[gp path] (1.688,1.970)--(1.778,1.970);
\draw[gp path] (7.829,1.970)--(7.739,1.970);
\draw[gp path] (1.688,2.039)--(1.868,2.039);
\draw[gp path] (7.829,2.039)--(7.649,2.039);
\node[gp node right] at (1.504,2.039) { 10};
\draw[gp path] (1.688,2.253)--(1.778,2.253);
\draw[gp path] (7.829,2.253)--(7.739,2.253);
\draw[gp path] (1.688,2.536)--(1.778,2.536);
\draw[gp path] (7.829,2.536)--(7.739,2.536);
\draw[gp path] (1.688,2.681)--(1.778,2.681);
\draw[gp path] (7.829,2.681)--(7.739,2.681);
\draw[gp path] (1.688,2.750)--(1.868,2.750);
\draw[gp path] (7.829,2.750)--(7.649,2.750);
\node[gp node right] at (1.504,2.750) { 100};
\draw[gp path] (1.688,2.964)--(1.778,2.964);
\draw[gp path] (7.829,2.964)--(7.739,2.964);
\draw[gp path] (1.688,3.248)--(1.778,3.248);
\draw[gp path] (7.829,3.248)--(7.739,3.248);
\draw[gp path] (1.688,3.393)--(1.778,3.393);
\draw[gp path] (7.829,3.393)--(7.739,3.393);
\draw[gp path] (1.688,3.462)--(1.868,3.462);
\draw[gp path] (7.829,3.462)--(7.649,3.462);
\node[gp node right] at (1.504,3.462) { 1000};
\draw[gp path] (2.712,0.616)--(2.712,0.796);
\draw[gp path] (2.712,3.587)--(2.712,3.407);
\node[gp node center] at (2.712,0.308) {summary};
\draw[gp path] (3.735,0.616)--(3.735,0.796);
\draw[gp path] (3.735,3.587)--(3.735,3.407);
\node[gp node center] at (3.735,0.308) {cor};
\draw[gp path] (4.759,0.616)--(4.759,0.796);
\draw[gp path] (4.759,3.587)--(4.759,3.407);
\node[gp node center] at (4.759,0.308) {SVD};
\draw[gp path] (5.782,0.616)--(5.782,0.796);
\draw[gp path] (5.782,3.587)--(5.782,3.407);
\node[gp node center] at (5.782,0.308) {KMeans};
\draw[gp path] (6.806,0.616)--(6.806,0.796);
\draw[gp path] (6.806,3.587)--(6.806,3.407);
\node[gp node center] at (6.806,0.308) {GMM};
\draw[gp path] (1.688,3.587)--(1.688,0.616)--(7.829,0.616)--(7.829,3.587)--cycle;
\node[gp node center,rotate=-270] at (0.246,2.101) {Runtime (s)};
\node[gp node right] at (2.372,4.238) {FM-IM};
\def\gpfillpath{(2.556,4.161)--(3.472,4.161)--(3.472,4.315)--(2.556,4.315)--cycle}
\gpfill{color=gpbgfillcolor} \gpfillpath;
\gpfill{color=gp lt color 0,gp pattern 0,pattern color=.} \gpfillpath;
\gpcolor{color=gp lt color 0}
\gpsetlinetype{gp lt plot 0}
\draw[gp path] (2.556,4.161)--(3.472,4.161)--(3.472,4.315)--(2.556,4.315)--cycle;
\def\gpfillpath{(2.507,0.616)--(2.713,0.616)--(2.713,1.955)--(2.507,1.955)--cycle}
\gpfill{color=gpbgfillcolor} \gpfillpath;
\gpfill{color=gp lt color 0,gp pattern 0,pattern color=.} \gpfillpath;
\draw[gp path] (2.507,0.616)--(2.507,1.954)--(2.712,1.954)--(2.712,0.616)--cycle;
\def\gpfillpath{(3.530,0.616)--(3.736,0.616)--(3.736,2.185)--(3.530,2.185)--cycle}
\gpfill{color=gpbgfillcolor} \gpfillpath;
\gpfill{color=gp lt color 0,gp pattern 0,pattern color=.} \gpfillpath;
\draw[gp path] (3.530,0.616)--(3.530,2.184)--(3.735,2.184)--(3.735,0.616)--cycle;
\def\gpfillpath{(4.554,0.616)--(4.760,0.616)--(4.760,2.125)--(4.554,2.125)--cycle}
\gpfill{color=gpbgfillcolor} \gpfillpath;
\gpfill{color=gp lt color 0,gp pattern 0,pattern color=.} \gpfillpath;
\draw[gp path] (4.554,0.616)--(4.554,2.124)--(4.759,2.124)--(4.759,0.616)--cycle;
\def\gpfillpath{(5.577,0.616)--(5.783,0.616)--(5.783,2.222)--(5.577,2.222)--cycle}
\gpfill{color=gpbgfillcolor} \gpfillpath;
\gpfill{color=gp lt color 0,gp pattern 0,pattern color=.} \gpfillpath;
\draw[gp path] (5.577,0.616)--(5.577,2.221)--(5.782,2.221)--(5.782,0.616)--cycle;
\def\gpfillpath{(6.601,0.616)--(6.807,0.616)--(6.807,3.191)--(6.601,3.191)--cycle}
\gpfill{color=gpbgfillcolor} \gpfillpath;
\gpfill{color=gp lt color 0,gp pattern 0,pattern color=.} \gpfillpath;
\draw[gp path] (6.601,0.616)--(6.601,3.190)--(6.806,3.190)--(6.806,0.616)--cycle;
\gpcolor{color=gp lt color border}
\node[gp node right] at (4.576,4.238) {FM-EM};
\def\gpfillpath{(4.760,4.161)--(5.676,4.161)--(5.676,4.315)--(4.760,4.315)--cycle}
\gpfill{color=gpbgfillcolor} \gpfillpath;
\gpfill{color=gp lt color 1,gp pattern 1,pattern color=.} \gpfillpath;
\gpcolor{color=gp lt color 1}
\gpsetlinetype{gp lt plot 1}
\draw[gp path] (4.760,4.161)--(5.676,4.161)--(5.676,4.315)--(4.760,4.315)--cycle;
\def\gpfillpath{(2.712,0.616)--(2.917,0.616)--(2.917,2.298)--(2.712,2.298)--cycle}
\gpfill{color=gpbgfillcolor} \gpfillpath;
\gpfill{color=gp lt color 1,gp pattern 1,pattern color=.} \gpfillpath;
\draw[gp path] (2.712,0.616)--(2.712,2.297)--(2.916,2.297)--(2.916,0.616)--cycle;
\def\gpfillpath{(3.735,0.616)--(3.941,0.616)--(3.941,2.298)--(3.735,2.298)--cycle}
\gpfill{color=gpbgfillcolor} \gpfillpath;
\gpfill{color=gp lt color 1,gp pattern 1,pattern color=.} \gpfillpath;
\draw[gp path] (3.735,0.616)--(3.735,2.297)--(3.940,2.297)--(3.940,0.616)--cycle;
\def\gpfillpath{(4.759,0.616)--(4.964,0.616)--(4.964,2.294)--(4.759,2.294)--cycle}
\gpfill{color=gpbgfillcolor} \gpfillpath;
\gpfill{color=gp lt color 1,gp pattern 1,pattern color=.} \gpfillpath;
\draw[gp path] (4.759,0.616)--(4.759,2.293)--(4.963,2.293)--(4.963,0.616)--cycle;
\def\gpfillpath{(5.782,0.616)--(5.988,0.616)--(5.988,2.302)--(5.782,2.302)--cycle}
\gpfill{color=gpbgfillcolor} \gpfillpath;
\gpfill{color=gp lt color 1,gp pattern 1,pattern color=.} \gpfillpath;
\draw[gp path] (5.782,0.616)--(5.782,2.301)--(5.987,2.301)--(5.987,0.616)--cycle;
\def\gpfillpath{(6.806,0.616)--(7.011,0.616)--(7.011,3.216)--(6.806,3.216)--cycle}
\gpfill{color=gpbgfillcolor} \gpfillpath;
\gpfill{color=gp lt color 1,gp pattern 1,pattern color=.} \gpfillpath;
\draw[gp path] (6.806,0.616)--(6.806,3.215)--(7.010,3.215)--(7.010,0.616)--cycle;
\gpcolor{color=gp lt color border}
\node[gp node right] at (6.780,4.238) {MLlib};
\def\gpfillpath{(6.964,4.161)--(7.880,4.161)--(7.880,4.315)--(6.964,4.315)--cycle}
\gpfill{color=gpbgfillcolor} \gpfillpath;
\gpfill{color=gp lt color 2,gp pattern 2,pattern color=.} \gpfillpath;
\gpcolor{color=gp lt color 2}
\gpsetlinetype{gp lt plot 2}
\draw[gp path] (6.964,4.161)--(7.880,4.161)--(7.880,4.315)--(6.964,4.315)--cycle;
\def\gpfillpath{(2.916,0.616)--(3.122,0.616)--(3.122,2.701)--(2.916,2.701)--cycle}
\gpfill{color=gpbgfillcolor} \gpfillpath;
\gpfill{color=gp lt color 2,gp pattern 2,pattern color=.} \gpfillpath;
\draw[gp path] (2.916,0.616)--(2.916,2.700)--(3.121,2.700)--(3.121,0.616)--cycle;
\def\gpfillpath{(3.940,0.616)--(4.145,0.616)--(4.145,3.006)--(3.940,3.006)--cycle}
\gpfill{color=gpbgfillcolor} \gpfillpath;
\gpfill{color=gp lt color 2,gp pattern 2,pattern color=.} \gpfillpath;
\draw[gp path] (3.940,0.616)--(3.940,3.005)--(4.144,3.005)--(4.144,0.616)--cycle;
\def\gpfillpath{(4.963,0.616)--(5.169,0.616)--(5.169,2.739)--(4.963,2.739)--cycle}
\gpfill{color=gpbgfillcolor} \gpfillpath;
\gpfill{color=gp lt color 2,gp pattern 2,pattern color=.} \gpfillpath;
\draw[gp path] (4.963,0.616)--(4.963,2.738)--(5.168,2.738)--(5.168,0.616)--cycle;
\def\gpfillpath{(5.987,0.616)--(6.192,0.616)--(6.192,2.875)--(5.987,2.875)--cycle}
\gpfill{color=gpbgfillcolor} \gpfillpath;
\gpfill{color=gp lt color 2,gp pattern 2,pattern color=.} \gpfillpath;
\draw[gp path] (5.987,0.616)--(5.987,2.874)--(6.191,2.874)--(6.191,0.616)--cycle;
\def\gpfillpath{(7.010,0.616)--(7.216,0.616)--(7.216,3.576)--(7.010,3.576)--cycle}
\gpfill{color=gpbgfillcolor} \gpfillpath;
\gpfill{color=gp lt color 2,gp pattern 2,pattern color=.} \gpfillpath;
\draw[gp path] (7.010,0.616)--(7.010,3.575)--(7.215,3.575)--(7.215,0.616)--cycle;
\gpcolor{color=gp lt color border}
\gpsetlinetype{gp lt border}
\draw[gp path] (1.688,3.587)--(1.688,0.616)--(7.829,0.616)--(7.829,3.587)--cycle;
%% coordinates of the plot area
\gpdefrectangularnode{gp plot 1}{\pgfpoint{1.688cm}{0.616cm}}{\pgfpoint{7.829cm}{3.587cm}}
\end{tikzpicture}
%% gnuplot variables

		\label{perf:rt}
		\vspace{-15pt}
		\caption{Runtime}
	\end{subfigure}

	%\vspace{-5pt}
	\begin{subfigure}{.5\textwidth}
		\begin{tikzpicture}[gnuplot]
%% generated with GNUPLOT 4.6p4 (Lua 5.1; terminal rev. 99, script rev. 100)
%% Fri 08 Apr 2016 10:23:19 PM EDT
\path (0.000,0.000) rectangle (8.382,4.572);
\gpcolor{color=gp lt color border}
\gpsetlinetype{gp lt border}
\gpsetlinewidth{1.00}
\draw[gp path] (1.504,0.616)--(1.684,0.616);
\draw[gp path] (7.829,0.616)--(7.649,0.616);
\node[gp node right] at (1.320,0.616) { 0};
\draw[gp path] (1.504,0.946)--(1.684,0.946);
\draw[gp path] (7.829,0.946)--(7.649,0.946);
\node[gp node right] at (1.320,0.946) { 50};
\draw[gp path] (1.504,1.276)--(1.684,1.276);
\draw[gp path] (7.829,1.276)--(7.649,1.276);
\node[gp node right] at (1.320,1.276) { 100};
\draw[gp path] (1.504,1.606)--(1.684,1.606);
\draw[gp path] (7.829,1.606)--(7.649,1.606);
\node[gp node right] at (1.320,1.606) { 150};
\draw[gp path] (1.504,1.936)--(1.684,1.936);
\draw[gp path] (7.829,1.936)--(7.649,1.936);
\node[gp node right] at (1.320,1.936) { 200};
\draw[gp path] (1.504,2.267)--(1.684,2.267);
\draw[gp path] (7.829,2.267)--(7.649,2.267);
\node[gp node right] at (1.320,2.267) { 250};
\draw[gp path] (1.504,2.597)--(1.684,2.597);
\draw[gp path] (7.829,2.597)--(7.649,2.597);
\node[gp node right] at (1.320,2.597) { 300};
\draw[gp path] (1.504,2.927)--(1.684,2.927);
\draw[gp path] (7.829,2.927)--(7.649,2.927);
\node[gp node right] at (1.320,2.927) { 350};
\draw[gp path] (1.504,3.257)--(1.684,3.257);
\draw[gp path] (7.829,3.257)--(7.649,3.257);
\node[gp node right] at (1.320,3.257) { 400};
\draw[gp path] (1.504,3.587)--(1.684,3.587);
\draw[gp path] (7.829,3.587)--(7.649,3.587);
\node[gp node right] at (1.320,3.587) { 450};
\draw[gp path] (2.558,0.616)--(2.558,0.796);
\draw[gp path] (2.558,3.587)--(2.558,3.407);
\node[gp node center] at (2.558,0.308) {summary};
\draw[gp path] (3.612,0.616)--(3.612,0.796);
\draw[gp path] (3.612,3.587)--(3.612,3.407);
\node[gp node center] at (3.612,0.308) {cor};
\draw[gp path] (4.667,0.616)--(4.667,0.796);
\draw[gp path] (4.667,3.587)--(4.667,3.407);
\node[gp node center] at (4.667,0.308) {SVD};
\draw[gp path] (5.721,0.616)--(5.721,0.796);
\draw[gp path] (5.721,3.587)--(5.721,3.407);
\node[gp node center] at (5.721,0.308) {KMeans};
\draw[gp path] (6.775,0.616)--(6.775,0.796);
\draw[gp path] (6.775,3.587)--(6.775,3.407);
\node[gp node center] at (6.775,0.308) {GMM};
\draw[gp path] (1.504,3.587)--(1.504,0.616)--(7.829,0.616)--(7.829,3.587)--cycle;
\node[gp node center,rotate=-270] at (0.246,2.101) {Memory (GB)};
\node[gp node right] at (2.280,4.238) {FM-IM};
\def\gpfillpath{(2.464,4.161)--(3.380,4.161)--(3.380,4.315)--(2.464,4.315)--cycle}
\gpfill{color=gpbgfillcolor} \gpfillpath;
\gpfill{color=gp lt color 0,gp pattern 0,pattern color=.} \gpfillpath;
\gpcolor{color=gp lt color 0}
\gpsetlinetype{gp lt plot 0}
\draw[gp path] (2.464,4.161)--(3.380,4.161)--(3.380,4.315)--(2.464,4.315)--cycle;
\def\gpfillpath{(2.347,0.616)--(2.559,0.616)--(2.559,2.267)--(2.347,2.267)--cycle}
\gpfill{color=gpbgfillcolor} \gpfillpath;
\gpfill{color=gp lt color 0,gp pattern 0,pattern color=.} \gpfillpath;
\draw[gp path] (2.347,0.616)--(2.347,2.266)--(2.558,2.266)--(2.558,0.616)--cycle;
\def\gpfillpath{(3.402,0.616)--(3.613,0.616)--(3.613,2.267)--(3.402,2.267)--cycle}
\gpfill{color=gpbgfillcolor} \gpfillpath;
\gpfill{color=gp lt color 0,gp pattern 0,pattern color=.} \gpfillpath;
\draw[gp path] (3.402,0.616)--(3.402,2.266)--(3.612,2.266)--(3.612,0.616)--cycle;
\def\gpfillpath{(4.456,0.616)--(4.668,0.616)--(4.668,2.267)--(4.456,2.267)--cycle}
\gpfill{color=gpbgfillcolor} \gpfillpath;
\gpfill{color=gp lt color 0,gp pattern 0,pattern color=.} \gpfillpath;
\draw[gp path] (4.456,0.616)--(4.456,2.266)--(4.667,2.266)--(4.667,0.616)--cycle;
\def\gpfillpath{(5.510,0.616)--(5.722,0.616)--(5.722,2.402)--(5.510,2.402)--cycle}
\gpfill{color=gpbgfillcolor} \gpfillpath;
\gpfill{color=gp lt color 0,gp pattern 0,pattern color=.} \gpfillpath;
\draw[gp path] (5.510,0.616)--(5.510,2.401)--(5.721,2.401)--(5.721,0.616)--cycle;
\def\gpfillpath{(6.564,0.616)--(6.776,0.616)--(6.776,3.238)--(6.564,3.238)--cycle}
\gpfill{color=gpbgfillcolor} \gpfillpath;
\gpfill{color=gp lt color 0,gp pattern 0,pattern color=.} \gpfillpath;
\draw[gp path] (6.564,0.616)--(6.564,3.237)--(6.775,3.237)--(6.775,0.616)--cycle;
\gpcolor{color=gp lt color border}
\node[gp node right] at (4.484,4.238) {FM-EM};
\def\gpfillpath{(4.668,4.161)--(5.584,4.161)--(5.584,4.315)--(4.668,4.315)--cycle}
\gpfill{color=gpbgfillcolor} \gpfillpath;
\gpfill{color=gp lt color 1,gp pattern 1,pattern color=.} \gpfillpath;
\gpcolor{color=gp lt color 1}
\gpsetlinetype{gp lt plot 1}
\draw[gp path] (4.668,4.161)--(5.584,4.161)--(5.584,4.315)--(4.668,4.315)--cycle;
\def\gpfillpath{(2.558,0.616)--(2.770,0.616)--(2.770,0.642)--(2.558,0.642)--cycle}
\gpfill{color=gpbgfillcolor} \gpfillpath;
\gpfill{color=gp lt color 1,gp pattern 1,pattern color=.} \gpfillpath;
\draw[gp path] (2.558,0.616)--(2.558,0.641)--(2.769,0.641)--(2.769,0.616)--cycle;
\def\gpfillpath{(3.612,0.616)--(3.824,0.616)--(3.824,0.643)--(3.612,0.643)--cycle}
\gpfill{color=gpbgfillcolor} \gpfillpath;
\gpfill{color=gp lt color 1,gp pattern 1,pattern color=.} \gpfillpath;
\draw[gp path] (3.612,0.616)--(3.612,0.642)--(3.823,0.642)--(3.823,0.616)--cycle;
\def\gpfillpath{(4.667,0.616)--(4.878,0.616)--(4.878,0.651)--(4.667,0.651)--cycle}
\gpfill{color=gpbgfillcolor} \gpfillpath;
\gpfill{color=gp lt color 1,gp pattern 1,pattern color=.} \gpfillpath;
\draw[gp path] (4.667,0.616)--(4.667,0.650)--(4.877,0.650)--(4.877,0.616)--cycle;
\def\gpfillpath{(5.721,0.616)--(5.933,0.616)--(5.933,0.696)--(5.721,0.696)--cycle}
\gpfill{color=gpbgfillcolor} \gpfillpath;
\gpfill{color=gp lt color 1,gp pattern 1,pattern color=.} \gpfillpath;
\draw[gp path] (5.721,0.616)--(5.721,0.695)--(5.932,0.695)--(5.932,0.616)--cycle;
\def\gpfillpath{(6.775,0.616)--(6.987,0.616)--(6.987,0.656)--(6.775,0.656)--cycle}
\gpfill{color=gpbgfillcolor} \gpfillpath;
\gpfill{color=gp lt color 1,gp pattern 1,pattern color=.} \gpfillpath;
\draw[gp path] (6.775,0.616)--(6.775,0.655)--(6.986,0.655)--(6.986,0.616)--cycle;
\gpcolor{color=gp lt color border}
\node[gp node right] at (6.688,4.238) {MLlib};
\def\gpfillpath{(6.872,4.161)--(7.788,4.161)--(7.788,4.315)--(6.872,4.315)--cycle}
\gpfill{color=gpbgfillcolor} \gpfillpath;
\gpfill{color=gp lt color 2,gp pattern 2,pattern color=.} \gpfillpath;
\gpcolor{color=gp lt color 2}
\gpsetlinetype{gp lt plot 2}
\draw[gp path] (6.872,4.161)--(7.788,4.161)--(7.788,4.315)--(6.872,4.315)--cycle;
\def\gpfillpath{(2.769,0.616)--(2.981,0.616)--(2.981,3.037)--(2.769,3.037)--cycle}
\gpfill{color=gpbgfillcolor} \gpfillpath;
\gpfill{color=gp lt color 2,gp pattern 2,pattern color=.} \gpfillpath;
\draw[gp path] (2.769,0.616)--(2.769,3.036)--(2.980,3.036)--(2.980,0.616)--cycle;
\def\gpfillpath{(3.823,0.616)--(4.035,0.616)--(4.035,3.037)--(3.823,3.037)--cycle}
\gpfill{color=gpbgfillcolor} \gpfillpath;
\gpfill{color=gp lt color 2,gp pattern 2,pattern color=.} \gpfillpath;
\draw[gp path] (3.823,0.616)--(3.823,3.036)--(4.034,3.036)--(4.034,0.616)--cycle;
\def\gpfillpath{(4.877,0.616)--(5.089,0.616)--(5.089,3.037)--(4.877,3.037)--cycle}
\gpfill{color=gpbgfillcolor} \gpfillpath;
\gpfill{color=gp lt color 2,gp pattern 2,pattern color=.} \gpfillpath;
\draw[gp path] (4.877,0.616)--(4.877,3.036)--(5.088,3.036)--(5.088,0.616)--cycle;
\def\gpfillpath{(5.932,0.616)--(6.143,0.616)--(6.143,3.390)--(5.932,3.390)--cycle}
\gpfill{color=gpbgfillcolor} \gpfillpath;
\gpfill{color=gp lt color 2,gp pattern 2,pattern color=.} \gpfillpath;
\draw[gp path] (5.932,0.616)--(5.932,3.389)--(6.142,3.389)--(6.142,0.616)--cycle;
\def\gpfillpath{(6.986,0.616)--(7.198,0.616)--(7.198,3.321)--(6.986,3.321)--cycle}
\gpfill{color=gpbgfillcolor} \gpfillpath;
\gpfill{color=gp lt color 2,gp pattern 2,pattern color=.} \gpfillpath;
\draw[gp path] (6.986,0.616)--(6.986,3.320)--(7.197,3.320)--(7.197,0.616)--cycle;
\gpcolor{color=gp lt color border}
\gpsetlinetype{gp lt border}
\draw[gp path] (1.504,3.587)--(1.504,0.616)--(7.829,0.616)--(7.829,3.587)--cycle;
%% coordinates of the plot area
\gpdefrectangularnode{gp plot 1}{\pgfpoint{1.504cm}{0.616cm}}{\pgfpoint{7.829cm}{3.587cm}}
\end{tikzpicture}
%% gnuplot variables

		\label{perf:mem}
		\vspace{-15pt}
		\caption{Memory consumption}
	\end{subfigure}
	\caption{The performance and memory consumption of FlashMatrix both
		in memory and on SSDs compared with Spark MLlib on a matrix with
	one billion rows and 32 columns.}
	\label{perf:fm}
\end{figure}

%\begin{figure}
%	\begin{center}
%		\footnotesize
%		\vspace{-15pt}
%		\begin{tikzpicture}[gnuplot]
%% generated with GNUPLOT 4.6p4 (Lua 5.1; terminal rev. 99, script rev. 100)
%% Fri 08 Apr 2016 06:45:40 PM EDT
\path (0.000,0.000) rectangle (8.382,4.572);
\gpcolor{color=gp lt color border}
\gpsetlinetype{gp lt border}
\gpsetlinewidth{1.00}
\draw[gp path] (1.688,0.616)--(1.868,0.616);
\draw[gp path] (7.829,0.616)--(7.649,0.616);
\node[gp node right] at (1.504,0.616) { 0.1};
\draw[gp path] (1.688,0.830)--(1.778,0.830);
\draw[gp path] (7.829,0.830)--(7.739,0.830);
\draw[gp path] (1.688,1.113)--(1.778,1.113);
\draw[gp path] (7.829,1.113)--(7.739,1.113);
\draw[gp path] (1.688,1.258)--(1.778,1.258);
\draw[gp path] (7.829,1.258)--(7.739,1.258);
\draw[gp path] (1.688,1.327)--(1.868,1.327);
\draw[gp path] (7.829,1.327)--(7.649,1.327);
\node[gp node right] at (1.504,1.327) { 1};
\draw[gp path] (1.688,1.542)--(1.778,1.542);
\draw[gp path] (7.829,1.542)--(7.739,1.542);
\draw[gp path] (1.688,1.825)--(1.778,1.825);
\draw[gp path] (7.829,1.825)--(7.739,1.825);
\draw[gp path] (1.688,1.970)--(1.778,1.970);
\draw[gp path] (7.829,1.970)--(7.739,1.970);
\draw[gp path] (1.688,2.039)--(1.868,2.039);
\draw[gp path] (7.829,2.039)--(7.649,2.039);
\node[gp node right] at (1.504,2.039) { 10};
\draw[gp path] (1.688,2.253)--(1.778,2.253);
\draw[gp path] (7.829,2.253)--(7.739,2.253);
\draw[gp path] (1.688,2.536)--(1.778,2.536);
\draw[gp path] (7.829,2.536)--(7.739,2.536);
\draw[gp path] (1.688,2.681)--(1.778,2.681);
\draw[gp path] (7.829,2.681)--(7.739,2.681);
\draw[gp path] (1.688,2.750)--(1.868,2.750);
\draw[gp path] (7.829,2.750)--(7.649,2.750);
\node[gp node right] at (1.504,2.750) { 100};
\draw[gp path] (1.688,2.964)--(1.778,2.964);
\draw[gp path] (7.829,2.964)--(7.739,2.964);
\draw[gp path] (1.688,3.248)--(1.778,3.248);
\draw[gp path] (7.829,3.248)--(7.739,3.248);
\draw[gp path] (1.688,3.393)--(1.778,3.393);
\draw[gp path] (7.829,3.393)--(7.739,3.393);
\draw[gp path] (1.688,3.462)--(1.868,3.462);
\draw[gp path] (7.829,3.462)--(7.649,3.462);
\node[gp node right] at (1.504,3.462) { 1000};
\draw[gp path] (2.712,0.616)--(2.712,0.796);
\draw[gp path] (2.712,3.587)--(2.712,3.407);
\node[gp node center] at (2.712,0.308) {summary};
\draw[gp path] (3.735,0.616)--(3.735,0.796);
\draw[gp path] (3.735,3.587)--(3.735,3.407);
\node[gp node center] at (3.735,0.308) {cor};
\draw[gp path] (4.759,0.616)--(4.759,0.796);
\draw[gp path] (4.759,3.587)--(4.759,3.407);
\node[gp node center] at (4.759,0.308) {SVD};
\draw[gp path] (5.782,0.616)--(5.782,0.796);
\draw[gp path] (5.782,3.587)--(5.782,3.407);
\node[gp node center] at (5.782,0.308) {KMeans};
\draw[gp path] (6.806,0.616)--(6.806,0.796);
\draw[gp path] (6.806,3.587)--(6.806,3.407);
\node[gp node center] at (6.806,0.308) {GMM};
\draw[gp path] (1.688,3.587)--(1.688,0.616)--(7.829,0.616)--(7.829,3.587)--cycle;
\node[gp node center,rotate=-270] at (0.246,2.101) {Runtime (s)};
\node[gp node right] at (2.372,4.238) {FM-IM};
\def\gpfillpath{(2.556,4.161)--(3.472,4.161)--(3.472,4.315)--(2.556,4.315)--cycle}
\gpfill{color=gpbgfillcolor} \gpfillpath;
\gpfill{color=gp lt color 0,gp pattern 0,pattern color=.} \gpfillpath;
\gpcolor{color=gp lt color 0}
\gpsetlinetype{gp lt plot 0}
\draw[gp path] (2.556,4.161)--(3.472,4.161)--(3.472,4.315)--(2.556,4.315)--cycle;
\def\gpfillpath{(2.507,0.616)--(2.713,0.616)--(2.713,1.955)--(2.507,1.955)--cycle}
\gpfill{color=gpbgfillcolor} \gpfillpath;
\gpfill{color=gp lt color 0,gp pattern 0,pattern color=.} \gpfillpath;
\draw[gp path] (2.507,0.616)--(2.507,1.954)--(2.712,1.954)--(2.712,0.616)--cycle;
\def\gpfillpath{(3.530,0.616)--(3.736,0.616)--(3.736,2.185)--(3.530,2.185)--cycle}
\gpfill{color=gpbgfillcolor} \gpfillpath;
\gpfill{color=gp lt color 0,gp pattern 0,pattern color=.} \gpfillpath;
\draw[gp path] (3.530,0.616)--(3.530,2.184)--(3.735,2.184)--(3.735,0.616)--cycle;
\def\gpfillpath{(4.554,0.616)--(4.760,0.616)--(4.760,2.125)--(4.554,2.125)--cycle}
\gpfill{color=gpbgfillcolor} \gpfillpath;
\gpfill{color=gp lt color 0,gp pattern 0,pattern color=.} \gpfillpath;
\draw[gp path] (4.554,0.616)--(4.554,2.124)--(4.759,2.124)--(4.759,0.616)--cycle;
\def\gpfillpath{(5.577,0.616)--(5.783,0.616)--(5.783,2.222)--(5.577,2.222)--cycle}
\gpfill{color=gpbgfillcolor} \gpfillpath;
\gpfill{color=gp lt color 0,gp pattern 0,pattern color=.} \gpfillpath;
\draw[gp path] (5.577,0.616)--(5.577,2.221)--(5.782,2.221)--(5.782,0.616)--cycle;
\def\gpfillpath{(6.601,0.616)--(6.807,0.616)--(6.807,3.191)--(6.601,3.191)--cycle}
\gpfill{color=gpbgfillcolor} \gpfillpath;
\gpfill{color=gp lt color 0,gp pattern 0,pattern color=.} \gpfillpath;
\draw[gp path] (6.601,0.616)--(6.601,3.190)--(6.806,3.190)--(6.806,0.616)--cycle;
\gpcolor{color=gp lt color border}
\node[gp node right] at (4.576,4.238) {FM-EM};
\def\gpfillpath{(4.760,4.161)--(5.676,4.161)--(5.676,4.315)--(4.760,4.315)--cycle}
\gpfill{color=gpbgfillcolor} \gpfillpath;
\gpfill{color=gp lt color 1,gp pattern 1,pattern color=.} \gpfillpath;
\gpcolor{color=gp lt color 1}
\gpsetlinetype{gp lt plot 1}
\draw[gp path] (4.760,4.161)--(5.676,4.161)--(5.676,4.315)--(4.760,4.315)--cycle;
\def\gpfillpath{(2.712,0.616)--(2.917,0.616)--(2.917,2.298)--(2.712,2.298)--cycle}
\gpfill{color=gpbgfillcolor} \gpfillpath;
\gpfill{color=gp lt color 1,gp pattern 1,pattern color=.} \gpfillpath;
\draw[gp path] (2.712,0.616)--(2.712,2.297)--(2.916,2.297)--(2.916,0.616)--cycle;
\def\gpfillpath{(3.735,0.616)--(3.941,0.616)--(3.941,2.298)--(3.735,2.298)--cycle}
\gpfill{color=gpbgfillcolor} \gpfillpath;
\gpfill{color=gp lt color 1,gp pattern 1,pattern color=.} \gpfillpath;
\draw[gp path] (3.735,0.616)--(3.735,2.297)--(3.940,2.297)--(3.940,0.616)--cycle;
\def\gpfillpath{(4.759,0.616)--(4.964,0.616)--(4.964,2.294)--(4.759,2.294)--cycle}
\gpfill{color=gpbgfillcolor} \gpfillpath;
\gpfill{color=gp lt color 1,gp pattern 1,pattern color=.} \gpfillpath;
\draw[gp path] (4.759,0.616)--(4.759,2.293)--(4.963,2.293)--(4.963,0.616)--cycle;
\def\gpfillpath{(5.782,0.616)--(5.988,0.616)--(5.988,2.302)--(5.782,2.302)--cycle}
\gpfill{color=gpbgfillcolor} \gpfillpath;
\gpfill{color=gp lt color 1,gp pattern 1,pattern color=.} \gpfillpath;
\draw[gp path] (5.782,0.616)--(5.782,2.301)--(5.987,2.301)--(5.987,0.616)--cycle;
\def\gpfillpath{(6.806,0.616)--(7.011,0.616)--(7.011,3.216)--(6.806,3.216)--cycle}
\gpfill{color=gpbgfillcolor} \gpfillpath;
\gpfill{color=gp lt color 1,gp pattern 1,pattern color=.} \gpfillpath;
\draw[gp path] (6.806,0.616)--(6.806,3.215)--(7.010,3.215)--(7.010,0.616)--cycle;
\gpcolor{color=gp lt color border}
\node[gp node right] at (6.780,4.238) {MLlib};
\def\gpfillpath{(6.964,4.161)--(7.880,4.161)--(7.880,4.315)--(6.964,4.315)--cycle}
\gpfill{color=gpbgfillcolor} \gpfillpath;
\gpfill{color=gp lt color 2,gp pattern 2,pattern color=.} \gpfillpath;
\gpcolor{color=gp lt color 2}
\gpsetlinetype{gp lt plot 2}
\draw[gp path] (6.964,4.161)--(7.880,4.161)--(7.880,4.315)--(6.964,4.315)--cycle;
\def\gpfillpath{(2.916,0.616)--(3.122,0.616)--(3.122,2.701)--(2.916,2.701)--cycle}
\gpfill{color=gpbgfillcolor} \gpfillpath;
\gpfill{color=gp lt color 2,gp pattern 2,pattern color=.} \gpfillpath;
\draw[gp path] (2.916,0.616)--(2.916,2.700)--(3.121,2.700)--(3.121,0.616)--cycle;
\def\gpfillpath{(3.940,0.616)--(4.145,0.616)--(4.145,3.006)--(3.940,3.006)--cycle}
\gpfill{color=gpbgfillcolor} \gpfillpath;
\gpfill{color=gp lt color 2,gp pattern 2,pattern color=.} \gpfillpath;
\draw[gp path] (3.940,0.616)--(3.940,3.005)--(4.144,3.005)--(4.144,0.616)--cycle;
\def\gpfillpath{(4.963,0.616)--(5.169,0.616)--(5.169,2.739)--(4.963,2.739)--cycle}
\gpfill{color=gpbgfillcolor} \gpfillpath;
\gpfill{color=gp lt color 2,gp pattern 2,pattern color=.} \gpfillpath;
\draw[gp path] (4.963,0.616)--(4.963,2.738)--(5.168,2.738)--(5.168,0.616)--cycle;
\def\gpfillpath{(5.987,0.616)--(6.192,0.616)--(6.192,2.875)--(5.987,2.875)--cycle}
\gpfill{color=gpbgfillcolor} \gpfillpath;
\gpfill{color=gp lt color 2,gp pattern 2,pattern color=.} \gpfillpath;
\draw[gp path] (5.987,0.616)--(5.987,2.874)--(6.191,2.874)--(6.191,0.616)--cycle;
\def\gpfillpath{(7.010,0.616)--(7.216,0.616)--(7.216,3.576)--(7.010,3.576)--cycle}
\gpfill{color=gpbgfillcolor} \gpfillpath;
\gpfill{color=gp lt color 2,gp pattern 2,pattern color=.} \gpfillpath;
\draw[gp path] (7.010,0.616)--(7.010,3.575)--(7.215,3.575)--(7.215,0.616)--cycle;
\gpcolor{color=gp lt color border}
\gpsetlinetype{gp lt border}
\draw[gp path] (1.688,3.587)--(1.688,0.616)--(7.829,0.616)--(7.829,3.587)--cycle;
%% coordinates of the plot area
\gpdefrectangularnode{gp plot 1}{\pgfpoint{1.688cm}{0.616cm}}{\pgfpoint{7.829cm}{3.587cm}}
\end{tikzpicture}
%% gnuplot variables

%		\vspace{-15pt}
%		\caption{The performance of FlashMatrix both in memory and on SSDs
%			compared with Spark MLlib on a matrix with one billion rows and
%		32 columns.}
%		\label{perf:fm}
%	\end{center}
%\end{figure}

FlashMatrix both in memory and on SSDs outperforms Spark MLlib significantly
in all applications (Figure \ref{perf:fm} (a)). For some applications
such as correlation, SVD and GMM, even though both FlashMatrix and MLlib
implementations heavily rely on BLAS for matrix multiplication, FlashMatrix
outperforms MLlib significantly owing to matrix operation fusion and two-level
partitioning, which reduce data movement between memory and CPU.
%SVD is the only application whose FlashMatrix
%implementation does not outperform the one in MLlib when being executed out of
%core because this application requires to read the entire matrix multiple times
%and output a large matrix.

%\begin{figure}
%	\begin{center}
%		\footnotesize
%		\vspace{-15pt}
%		\begin{tikzpicture}[gnuplot]
%% generated with GNUPLOT 4.6p4 (Lua 5.1; terminal rev. 99, script rev. 100)
%% Fri 08 Apr 2016 10:23:19 PM EDT
\path (0.000,0.000) rectangle (8.382,4.572);
\gpcolor{color=gp lt color border}
\gpsetlinetype{gp lt border}
\gpsetlinewidth{1.00}
\draw[gp path] (1.504,0.616)--(1.684,0.616);
\draw[gp path] (7.829,0.616)--(7.649,0.616);
\node[gp node right] at (1.320,0.616) { 0};
\draw[gp path] (1.504,0.946)--(1.684,0.946);
\draw[gp path] (7.829,0.946)--(7.649,0.946);
\node[gp node right] at (1.320,0.946) { 50};
\draw[gp path] (1.504,1.276)--(1.684,1.276);
\draw[gp path] (7.829,1.276)--(7.649,1.276);
\node[gp node right] at (1.320,1.276) { 100};
\draw[gp path] (1.504,1.606)--(1.684,1.606);
\draw[gp path] (7.829,1.606)--(7.649,1.606);
\node[gp node right] at (1.320,1.606) { 150};
\draw[gp path] (1.504,1.936)--(1.684,1.936);
\draw[gp path] (7.829,1.936)--(7.649,1.936);
\node[gp node right] at (1.320,1.936) { 200};
\draw[gp path] (1.504,2.267)--(1.684,2.267);
\draw[gp path] (7.829,2.267)--(7.649,2.267);
\node[gp node right] at (1.320,2.267) { 250};
\draw[gp path] (1.504,2.597)--(1.684,2.597);
\draw[gp path] (7.829,2.597)--(7.649,2.597);
\node[gp node right] at (1.320,2.597) { 300};
\draw[gp path] (1.504,2.927)--(1.684,2.927);
\draw[gp path] (7.829,2.927)--(7.649,2.927);
\node[gp node right] at (1.320,2.927) { 350};
\draw[gp path] (1.504,3.257)--(1.684,3.257);
\draw[gp path] (7.829,3.257)--(7.649,3.257);
\node[gp node right] at (1.320,3.257) { 400};
\draw[gp path] (1.504,3.587)--(1.684,3.587);
\draw[gp path] (7.829,3.587)--(7.649,3.587);
\node[gp node right] at (1.320,3.587) { 450};
\draw[gp path] (2.558,0.616)--(2.558,0.796);
\draw[gp path] (2.558,3.587)--(2.558,3.407);
\node[gp node center] at (2.558,0.308) {summary};
\draw[gp path] (3.612,0.616)--(3.612,0.796);
\draw[gp path] (3.612,3.587)--(3.612,3.407);
\node[gp node center] at (3.612,0.308) {cor};
\draw[gp path] (4.667,0.616)--(4.667,0.796);
\draw[gp path] (4.667,3.587)--(4.667,3.407);
\node[gp node center] at (4.667,0.308) {SVD};
\draw[gp path] (5.721,0.616)--(5.721,0.796);
\draw[gp path] (5.721,3.587)--(5.721,3.407);
\node[gp node center] at (5.721,0.308) {KMeans};
\draw[gp path] (6.775,0.616)--(6.775,0.796);
\draw[gp path] (6.775,3.587)--(6.775,3.407);
\node[gp node center] at (6.775,0.308) {GMM};
\draw[gp path] (1.504,3.587)--(1.504,0.616)--(7.829,0.616)--(7.829,3.587)--cycle;
\node[gp node center,rotate=-270] at (0.246,2.101) {Memory (GB)};
\node[gp node right] at (2.280,4.238) {FM-IM};
\def\gpfillpath{(2.464,4.161)--(3.380,4.161)--(3.380,4.315)--(2.464,4.315)--cycle}
\gpfill{color=gpbgfillcolor} \gpfillpath;
\gpfill{color=gp lt color 0,gp pattern 0,pattern color=.} \gpfillpath;
\gpcolor{color=gp lt color 0}
\gpsetlinetype{gp lt plot 0}
\draw[gp path] (2.464,4.161)--(3.380,4.161)--(3.380,4.315)--(2.464,4.315)--cycle;
\def\gpfillpath{(2.347,0.616)--(2.559,0.616)--(2.559,2.267)--(2.347,2.267)--cycle}
\gpfill{color=gpbgfillcolor} \gpfillpath;
\gpfill{color=gp lt color 0,gp pattern 0,pattern color=.} \gpfillpath;
\draw[gp path] (2.347,0.616)--(2.347,2.266)--(2.558,2.266)--(2.558,0.616)--cycle;
\def\gpfillpath{(3.402,0.616)--(3.613,0.616)--(3.613,2.267)--(3.402,2.267)--cycle}
\gpfill{color=gpbgfillcolor} \gpfillpath;
\gpfill{color=gp lt color 0,gp pattern 0,pattern color=.} \gpfillpath;
\draw[gp path] (3.402,0.616)--(3.402,2.266)--(3.612,2.266)--(3.612,0.616)--cycle;
\def\gpfillpath{(4.456,0.616)--(4.668,0.616)--(4.668,2.267)--(4.456,2.267)--cycle}
\gpfill{color=gpbgfillcolor} \gpfillpath;
\gpfill{color=gp lt color 0,gp pattern 0,pattern color=.} \gpfillpath;
\draw[gp path] (4.456,0.616)--(4.456,2.266)--(4.667,2.266)--(4.667,0.616)--cycle;
\def\gpfillpath{(5.510,0.616)--(5.722,0.616)--(5.722,2.402)--(5.510,2.402)--cycle}
\gpfill{color=gpbgfillcolor} \gpfillpath;
\gpfill{color=gp lt color 0,gp pattern 0,pattern color=.} \gpfillpath;
\draw[gp path] (5.510,0.616)--(5.510,2.401)--(5.721,2.401)--(5.721,0.616)--cycle;
\def\gpfillpath{(6.564,0.616)--(6.776,0.616)--(6.776,3.238)--(6.564,3.238)--cycle}
\gpfill{color=gpbgfillcolor} \gpfillpath;
\gpfill{color=gp lt color 0,gp pattern 0,pattern color=.} \gpfillpath;
\draw[gp path] (6.564,0.616)--(6.564,3.237)--(6.775,3.237)--(6.775,0.616)--cycle;
\gpcolor{color=gp lt color border}
\node[gp node right] at (4.484,4.238) {FM-EM};
\def\gpfillpath{(4.668,4.161)--(5.584,4.161)--(5.584,4.315)--(4.668,4.315)--cycle}
\gpfill{color=gpbgfillcolor} \gpfillpath;
\gpfill{color=gp lt color 1,gp pattern 1,pattern color=.} \gpfillpath;
\gpcolor{color=gp lt color 1}
\gpsetlinetype{gp lt plot 1}
\draw[gp path] (4.668,4.161)--(5.584,4.161)--(5.584,4.315)--(4.668,4.315)--cycle;
\def\gpfillpath{(2.558,0.616)--(2.770,0.616)--(2.770,0.642)--(2.558,0.642)--cycle}
\gpfill{color=gpbgfillcolor} \gpfillpath;
\gpfill{color=gp lt color 1,gp pattern 1,pattern color=.} \gpfillpath;
\draw[gp path] (2.558,0.616)--(2.558,0.641)--(2.769,0.641)--(2.769,0.616)--cycle;
\def\gpfillpath{(3.612,0.616)--(3.824,0.616)--(3.824,0.643)--(3.612,0.643)--cycle}
\gpfill{color=gpbgfillcolor} \gpfillpath;
\gpfill{color=gp lt color 1,gp pattern 1,pattern color=.} \gpfillpath;
\draw[gp path] (3.612,0.616)--(3.612,0.642)--(3.823,0.642)--(3.823,0.616)--cycle;
\def\gpfillpath{(4.667,0.616)--(4.878,0.616)--(4.878,0.651)--(4.667,0.651)--cycle}
\gpfill{color=gpbgfillcolor} \gpfillpath;
\gpfill{color=gp lt color 1,gp pattern 1,pattern color=.} \gpfillpath;
\draw[gp path] (4.667,0.616)--(4.667,0.650)--(4.877,0.650)--(4.877,0.616)--cycle;
\def\gpfillpath{(5.721,0.616)--(5.933,0.616)--(5.933,0.696)--(5.721,0.696)--cycle}
\gpfill{color=gpbgfillcolor} \gpfillpath;
\gpfill{color=gp lt color 1,gp pattern 1,pattern color=.} \gpfillpath;
\draw[gp path] (5.721,0.616)--(5.721,0.695)--(5.932,0.695)--(5.932,0.616)--cycle;
\def\gpfillpath{(6.775,0.616)--(6.987,0.616)--(6.987,0.656)--(6.775,0.656)--cycle}
\gpfill{color=gpbgfillcolor} \gpfillpath;
\gpfill{color=gp lt color 1,gp pattern 1,pattern color=.} \gpfillpath;
\draw[gp path] (6.775,0.616)--(6.775,0.655)--(6.986,0.655)--(6.986,0.616)--cycle;
\gpcolor{color=gp lt color border}
\node[gp node right] at (6.688,4.238) {MLlib};
\def\gpfillpath{(6.872,4.161)--(7.788,4.161)--(7.788,4.315)--(6.872,4.315)--cycle}
\gpfill{color=gpbgfillcolor} \gpfillpath;
\gpfill{color=gp lt color 2,gp pattern 2,pattern color=.} \gpfillpath;
\gpcolor{color=gp lt color 2}
\gpsetlinetype{gp lt plot 2}
\draw[gp path] (6.872,4.161)--(7.788,4.161)--(7.788,4.315)--(6.872,4.315)--cycle;
\def\gpfillpath{(2.769,0.616)--(2.981,0.616)--(2.981,3.037)--(2.769,3.037)--cycle}
\gpfill{color=gpbgfillcolor} \gpfillpath;
\gpfill{color=gp lt color 2,gp pattern 2,pattern color=.} \gpfillpath;
\draw[gp path] (2.769,0.616)--(2.769,3.036)--(2.980,3.036)--(2.980,0.616)--cycle;
\def\gpfillpath{(3.823,0.616)--(4.035,0.616)--(4.035,3.037)--(3.823,3.037)--cycle}
\gpfill{color=gpbgfillcolor} \gpfillpath;
\gpfill{color=gp lt color 2,gp pattern 2,pattern color=.} \gpfillpath;
\draw[gp path] (3.823,0.616)--(3.823,3.036)--(4.034,3.036)--(4.034,0.616)--cycle;
\def\gpfillpath{(4.877,0.616)--(5.089,0.616)--(5.089,3.037)--(4.877,3.037)--cycle}
\gpfill{color=gpbgfillcolor} \gpfillpath;
\gpfill{color=gp lt color 2,gp pattern 2,pattern color=.} \gpfillpath;
\draw[gp path] (4.877,0.616)--(4.877,3.036)--(5.088,3.036)--(5.088,0.616)--cycle;
\def\gpfillpath{(5.932,0.616)--(6.143,0.616)--(6.143,3.390)--(5.932,3.390)--cycle}
\gpfill{color=gpbgfillcolor} \gpfillpath;
\gpfill{color=gp lt color 2,gp pattern 2,pattern color=.} \gpfillpath;
\draw[gp path] (5.932,0.616)--(5.932,3.389)--(6.142,3.389)--(6.142,0.616)--cycle;
\def\gpfillpath{(6.986,0.616)--(7.198,0.616)--(7.198,3.321)--(6.986,3.321)--cycle}
\gpfill{color=gpbgfillcolor} \gpfillpath;
\gpfill{color=gp lt color 2,gp pattern 2,pattern color=.} \gpfillpath;
\draw[gp path] (6.986,0.616)--(6.986,3.320)--(7.197,3.320)--(7.197,0.616)--cycle;
\gpcolor{color=gp lt color border}
\gpsetlinetype{gp lt border}
\draw[gp path] (1.504,3.587)--(1.504,0.616)--(7.829,0.616)--(7.829,3.587)--cycle;
%% coordinates of the plot area
\gpdefrectangularnode{gp plot 1}{\pgfpoint{1.504cm}{0.616cm}}{\pgfpoint{7.829cm}{3.587cm}}
\end{tikzpicture}
%% gnuplot variables

%		\vspace{-15pt}
%		\caption{The memory consumption of FlashMatrix in memory and on SSDs
%		as well as Spark MLlib on a matrix with one billion rows and 32 columns.}
%		\label{perf:mem}
%	\end{center}
%\end{figure}

Even though FlashMatrix provides a matrix-oriented functional programming
interface, the in-memory execution barely increases memory consumption from
the minimum memory requirement of the applications and the out-of-core
execution uses a very small memory footprint (Figure \ref{perf:fm} (b)).
The functional programming interface generates a new matrix in each matrix
operation and potentially leads to high memory consumption. Lazy evaluation
in FlashMatrix avoids materializing large matrices and significantly
reduces its memory consumption.
%The input matrix of the applications requires memory storage of 256GB and
%both SVD and GMM need extra memory of 80GB to store the output matrix.
When being executed on SSDs, FlashMatrix keeps both the input matrix and
the output matrix of the applications on SSDs and thus leads to a very small
memory footprint.

\begin{figure}
	\begin{center}
		\footnotesize
		\vspace{-15pt}
		\begin{tikzpicture}[gnuplot]
%% generated with GNUPLOT 4.6p4 (Lua 5.1; terminal rev. 99, script rev. 100)
%% Sat 02 Apr 2016 10:17:07 PM EDT
\path (0.000,0.000) rectangle (8.382,4.572);
\gpcolor{color=gp lt color border}
\gpsetlinetype{gp lt border}
\gpsetlinewidth{1.00}
\draw[gp path] (1.504,0.616)--(1.684,0.616);
\draw[gp path] (7.829,0.616)--(7.649,0.616);
\node[gp node right] at (1.320,0.616) { 0};
\draw[gp path] (1.504,1.291)--(1.684,1.291);
\draw[gp path] (7.829,1.291)--(7.649,1.291);
\node[gp node right] at (1.320,1.291) { 50};
\draw[gp path] (1.504,1.966)--(1.684,1.966);
\draw[gp path] (7.829,1.966)--(7.649,1.966);
\node[gp node right] at (1.320,1.966) { 100};
\draw[gp path] (1.504,2.642)--(1.684,2.642);
\draw[gp path] (7.829,2.642)--(7.649,2.642);
\node[gp node right] at (1.320,2.642) { 150};
\draw[gp path] (1.504,3.317)--(1.684,3.317);
\draw[gp path] (7.829,3.317)--(7.649,3.317);
\node[gp node right] at (1.320,3.317) { 200};
\draw[gp path] (2.769,0.616)--(2.769,0.796);
\draw[gp path] (2.769,3.587)--(2.769,3.407);
\node[gp node center] at (2.769,0.308) {summary};
\draw[gp path] (4.034,0.616)--(4.034,0.796);
\draw[gp path] (4.034,3.587)--(4.034,3.407);
\node[gp node center] at (4.034,0.308) {cor};
\draw[gp path] (5.299,0.616)--(5.299,0.796);
\draw[gp path] (5.299,3.587)--(5.299,3.407);
\node[gp node center] at (5.299,0.308) {SVD};
\draw[gp path] (6.564,0.616)--(6.564,0.796);
\draw[gp path] (6.564,3.587)--(6.564,3.407);
\node[gp node center] at (6.564,0.308) {KMeans};
\draw[gp path] (1.504,3.587)--(1.504,0.616)--(7.829,0.616)--(7.829,3.587)--cycle;
\node[gp node center,rotate=-270] at (0.246,2.101) {Runtime (s)};
\node[gp node right] at (3.382,4.238) {FM-IM};
\def\gpfillpath{(3.566,4.161)--(4.482,4.161)--(4.482,4.315)--(3.566,4.315)--cycle}
\gpfill{color=gpbgfillcolor} \gpfillpath;
\gpfill{color=gp lt color 0,gp pattern 0,pattern color=.} \gpfillpath;
\gpcolor{color=gp lt color 0}
\gpsetlinetype{gp lt plot 0}
\draw[gp path] (3.566,4.161)--(4.482,4.161)--(4.482,4.315)--(3.566,4.315)--cycle;
\def\gpfillpath{(2.611,0.616)--(2.928,0.616)--(2.928,0.862)--(2.611,0.862)--cycle}
\gpfill{color=gpbgfillcolor} \gpfillpath;
\gpfill{color=gp lt color 0,gp pattern 0,pattern color=.} \gpfillpath;
\draw[gp path] (2.611,0.616)--(2.611,0.861)--(2.927,0.861)--(2.927,0.616)--cycle;
\def\gpfillpath{(3.876,0.616)--(4.193,0.616)--(4.193,1.150)--(3.876,1.150)--cycle}
\gpfill{color=gpbgfillcolor} \gpfillpath;
\gpfill{color=gp lt color 0,gp pattern 0,pattern color=.} \gpfillpath;
\draw[gp path] (3.876,0.616)--(3.876,1.149)--(4.192,1.149)--(4.192,0.616)--cycle;
\def\gpfillpath{(5.141,0.616)--(5.458,0.616)--(5.458,1.063)--(5.141,1.063)--cycle}
\gpfill{color=gpbgfillcolor} \gpfillpath;
\gpfill{color=gp lt color 0,gp pattern 0,pattern color=.} \gpfillpath;
\draw[gp path] (5.141,0.616)--(5.141,1.062)--(5.457,1.062)--(5.457,0.616)--cycle;
\def\gpfillpath{(6.406,0.616)--(6.723,0.616)--(6.723,1.326)--(6.406,1.326)--cycle}
\gpfill{color=gpbgfillcolor} \gpfillpath;
\gpfill{color=gp lt color 0,gp pattern 0,pattern color=.} \gpfillpath;
\draw[gp path] (6.406,0.616)--(6.406,1.325)--(6.722,1.325)--(6.722,0.616)--cycle;
\gpcolor{color=gp lt color border}
\node[gp node right] at (5.586,4.238) {R};
\def\gpfillpath{(5.770,4.161)--(6.686,4.161)--(6.686,4.315)--(5.770,4.315)--cycle}
\gpfill{color=gpbgfillcolor} \gpfillpath;
\gpfill{color=gp lt color 1,gp pattern 1,pattern color=.} \gpfillpath;
\gpcolor{color=gp lt color 1}
\gpsetlinetype{gp lt plot 1}
\draw[gp path] (5.770,4.161)--(6.686,4.161)--(6.686,4.315)--(5.770,4.315)--cycle;
\def\gpfillpath{(2.927,0.616)--(3.244,0.616)--(3.244,1.146)--(2.927,1.146)--cycle}
\gpfill{color=gpbgfillcolor} \gpfillpath;
\gpfill{color=gp lt color 1,gp pattern 1,pattern color=.} \gpfillpath;
\draw[gp path] (2.927,0.616)--(2.927,1.145)--(3.243,1.145)--(3.243,0.616)--cycle;
\def\gpfillpath{(4.192,0.616)--(4.509,0.616)--(4.509,1.478)--(4.192,1.478)--cycle}
\gpfill{color=gpbgfillcolor} \gpfillpath;
\gpfill{color=gp lt color 1,gp pattern 1,pattern color=.} \gpfillpath;
\draw[gp path] (4.192,0.616)--(4.192,1.477)--(4.508,1.477)--(4.508,0.616)--cycle;
\def\gpfillpath{(5.457,0.616)--(5.774,0.616)--(5.774,3.534)--(5.457,3.534)--cycle}
\gpfill{color=gpbgfillcolor} \gpfillpath;
\gpfill{color=gp lt color 1,gp pattern 1,pattern color=.} \gpfillpath;
\draw[gp path] (5.457,0.616)--(5.457,3.533)--(5.773,3.533)--(5.773,0.616)--cycle;
\def\gpfillpath{(6.722,0.616)--(7.039,0.616)--(7.039,3.048)--(6.722,3.048)--cycle}
\gpfill{color=gpbgfillcolor} \gpfillpath;
\gpfill{color=gp lt color 1,gp pattern 1,pattern color=.} \gpfillpath;
\draw[gp path] (6.722,0.616)--(6.722,3.047)--(7.038,3.047)--(7.038,0.616)--cycle;
\gpcolor{color=gp lt color border}
\gpsetlinetype{gp lt border}
\draw[gp path] (1.504,3.587)--(1.504,0.616)--(7.829,0.616)--(7.829,3.587)--cycle;
%% coordinates of the plot area
\gpdefrectangularnode{gp plot 1}{\pgfpoint{1.504cm}{0.616cm}}{\pgfpoint{7.829cm}{3.587cm}}
\end{tikzpicture}
%% gnuplot variables

		\vspace{-10pt}
		\caption{The performance of FlashMatrix in a single thread both in
			memory and on SSDs compared with the C and FORTRAN implementations
		in the R framework.}
		\label{fig:fmR}
	\end{center}
\end{figure}

FlashMatrix running both in memory and on SSDs significantly outperforms R
even with a single thread in all of these applications (Figure \ref{fig:fmR}).
We exclude statistic summary in the experiment because R does not provide
a C or FORTRAN implementation of computing the same statistics. The performance
result indicates that FlashMatrix executes R code efficiently to even outperform
some optimized C and FORTRAN implementations when processing large datasets.

\begin{figure}
	\begin{center}
		\footnotesize
		\vspace{-15pt}
		\begin{tikzpicture}[gnuplot]
%% generated with GNUPLOT 4.6p4 (Lua 5.1; terminal rev. 99, script rev. 100)
%% Wed 06 Apr 2016 02:12:53 PM EDT
\path (0.000,0.000) rectangle (8.382,6.350);
\gpcolor{color=gp lt color border}
\gpsetlinetype{gp lt border}
\gpsetlinewidth{1.00}
\draw[gp path] (1.012,0.985)--(1.192,0.985);
\draw[gp path] (7.829,0.985)--(7.649,0.985);
\node[gp node right] at (0.828,0.985) { 1};
\draw[gp path] (1.012,1.549)--(1.192,1.549);
\draw[gp path] (7.829,1.549)--(7.649,1.549);
\node[gp node right] at (0.828,1.549) { 2};
\draw[gp path] (1.012,2.112)--(1.192,2.112);
\draw[gp path] (7.829,2.112)--(7.649,2.112);
\node[gp node right] at (0.828,2.112) { 4};
\draw[gp path] (1.012,2.676)--(1.192,2.676);
\draw[gp path] (7.829,2.676)--(7.649,2.676);
\node[gp node right] at (0.828,2.676) { 8};
\draw[gp path] (1.012,3.240)--(1.192,3.240);
\draw[gp path] (7.829,3.240)--(7.649,3.240);
\node[gp node right] at (0.828,3.240) { 16};
\draw[gp path] (1.012,3.803)--(1.192,3.803);
\draw[gp path] (7.829,3.803)--(7.649,3.803);
\node[gp node right] at (0.828,3.803) { 32};
\draw[gp path] (1.012,0.985)--(1.012,1.165);
\draw[gp path] (1.012,4.133)--(1.012,3.953);
\node[gp node center] at (1.012,0.677) {1};
\draw[gp path] (2.148,0.985)--(2.148,1.165);
\draw[gp path] (2.148,4.133)--(2.148,3.953);
\node[gp node center] at (2.148,0.677) {2};
\draw[gp path] (3.284,0.985)--(3.284,1.165);
\draw[gp path] (3.284,4.133)--(3.284,3.953);
\node[gp node center] at (3.284,0.677) {4};
\draw[gp path] (4.421,0.985)--(4.421,1.165);
\draw[gp path] (4.421,4.133)--(4.421,3.953);
\node[gp node center] at (4.421,0.677) {8};
\draw[gp path] (5.557,0.985)--(5.557,1.165);
\draw[gp path] (5.557,4.133)--(5.557,3.953);
\node[gp node center] at (5.557,0.677) {16};
\draw[gp path] (6.693,0.985)--(6.693,1.165);
\draw[gp path] (6.693,4.133)--(6.693,3.953);
\node[gp node center] at (6.693,0.677) {32};
\draw[gp path] (7.829,0.985)--(7.829,1.165);
\draw[gp path] (7.829,4.133)--(7.829,3.953);
\node[gp node center] at (7.829,0.677) {64};
\draw[gp path] (1.012,4.133)--(1.012,0.985)--(7.829,0.985)--(7.829,4.133)--cycle;
\node[gp node center] at (4.420,0.215) {The number of threads};
\node[gp node right] at (3.136,6.016) {summary-IM};
\gpcolor{color=gp lt color 0}
\gpsetlinetype{gp lt plot 0}
\draw[gp path] (3.320,6.016)--(4.236,6.016);
\draw[gp path] (1.012,0.985)--(2.148,1.520)--(3.284,2.096)--(4.421,2.620)--(5.557,3.160)%
  --(6.693,3.703)--(7.829,4.058);
\gpcolor{color=gp lt color border}
\node[gp node right] at (3.136,5.708) {cor-IM};
\gpcolor{color=gp lt color 1}
\gpsetlinetype{gp lt plot 1}
\draw[gp path] (3.320,5.708)--(4.236,5.708);
\draw[gp path] (1.012,0.985)--(2.148,1.536)--(3.284,2.130)--(4.421,2.650)--(5.557,3.197)%
  --(6.693,3.749)--(7.829,4.096);
\gpcolor{color=gp lt color border}
\node[gp node right] at (3.136,5.400) {SVD-IM};
\gpcolor{color=gp lt color 2}
\gpsetlinetype{gp lt plot 2}
\draw[gp path] (3.320,5.400)--(4.236,5.400);
\draw[gp path] (1.012,0.985)--(2.148,1.557)--(3.284,2.110)--(4.421,2.655)--(5.557,3.198)%
  --(6.693,3.748)--(7.829,4.089);
\gpcolor{color=gp lt color border}
\node[gp node right] at (3.136,5.092) {KMeans-IM};
\gpcolor{color=gp lt color 3}
\gpsetlinetype{gp lt plot 3}
\draw[gp path] (3.320,5.092)--(4.236,5.092);
\draw[gp path] (1.012,0.985)--(2.148,1.565)--(3.284,2.128)--(4.421,2.657)--(5.557,3.180)%
  --(6.693,3.711)--(7.829,4.071);
\gpcolor{color=gp lt color border}
\node[gp node right] at (3.136,4.784) {GMM-IM};
\gpcolor{color=gp lt color 4}
\gpsetlinetype{gp lt plot 4}
\draw[gp path] (3.320,4.784)--(4.236,4.784);
\draw[gp path] (1.012,0.985)--(2.148,1.540)--(3.284,2.110)--(4.421,2.654)--(5.557,3.197)%
  --(6.693,3.735)--(7.829,4.078);
\gpcolor{color=gp lt color border}
\node[gp node right] at (6.260,6.016) {summary-EM};
\gpcolor{color=gp lt color 5}
\gpsetlinetype{gp lt plot 5}
\draw[gp path] (6.444,6.016)--(7.360,6.016);
\draw[gp path] (1.012,0.985)--(2.148,1.549)--(3.284,1.992)--(4.421,2.591)--(5.557,3.112)%
  --(6.693,3.337)--(7.829,3.400);
\gpsetpointsize{4.00}
\gppoint{gp mark 6}{(1.012,0.985)}
\gppoint{gp mark 6}{(2.148,1.549)}
\gppoint{gp mark 6}{(3.284,1.992)}
\gppoint{gp mark 6}{(4.421,2.591)}
\gppoint{gp mark 6}{(5.557,3.112)}
\gppoint{gp mark 6}{(6.693,3.337)}
\gppoint{gp mark 6}{(7.829,3.400)}
\gppoint{gp mark 6}{(6.902,6.016)}
\gpcolor{color=gp lt color border}
\node[gp node right] at (6.260,5.708) {cor-EM};
\gpcolor{color=gp lt color 6}
\gpsetlinetype{gp lt plot 6}
\draw[gp path] (6.444,5.708)--(7.360,5.708);
\draw[gp path] (1.012,0.985)--(2.148,1.549)--(3.284,2.092)--(4.421,2.618)--(5.557,3.147)%
  --(6.693,3.605)--(7.829,3.629);
\gppoint{gp mark 7}{(1.012,0.985)}
\gppoint{gp mark 7}{(2.148,1.549)}
\gppoint{gp mark 7}{(3.284,2.092)}
\gppoint{gp mark 7}{(4.421,2.618)}
\gppoint{gp mark 7}{(5.557,3.147)}
\gppoint{gp mark 7}{(6.693,3.605)}
\gppoint{gp mark 7}{(7.829,3.629)}
\gppoint{gp mark 7}{(6.902,5.708)}
\gpcolor{color=gp lt color border}
\node[gp node right] at (6.260,5.400) {SVD-EM};
\gpcolor{color=gp lt color 7}
\gpsetlinetype{gp lt plot 7}
\draw[gp path] (6.444,5.400)--(7.360,5.400);
\draw[gp path] (1.012,0.985)--(2.148,1.545)--(3.284,2.085)--(4.421,2.606)--(5.557,3.126)%
  --(6.693,3.424)--(7.829,3.461);
\gppoint{gp mark 8}{(1.012,0.985)}
\gppoint{gp mark 8}{(2.148,1.545)}
\gppoint{gp mark 8}{(3.284,2.085)}
\gppoint{gp mark 8}{(4.421,2.606)}
\gppoint{gp mark 8}{(5.557,3.126)}
\gppoint{gp mark 8}{(6.693,3.424)}
\gppoint{gp mark 8}{(7.829,3.461)}
\gppoint{gp mark 8}{(6.902,5.400)}
\gpcolor{color=gp lt color border}
\node[gp node right] at (6.260,5.092) {KMeans-EM};
\gpcolor{color=gp lt color 0}
\gpsetlinetype{gp lt plot 0}
\draw[gp path] (6.444,5.092)--(7.360,5.092);
\draw[gp path] (1.012,0.985)--(2.148,1.549)--(3.284,2.094)--(4.421,2.597)--(5.557,3.087)%
  --(6.693,3.497)--(7.829,3.615);
\gppoint{gp mark 9}{(1.012,0.985)}
\gppoint{gp mark 9}{(2.148,1.549)}
\gppoint{gp mark 9}{(3.284,2.094)}
\gppoint{gp mark 9}{(4.421,2.597)}
\gppoint{gp mark 9}{(5.557,3.087)}
\gppoint{gp mark 9}{(6.693,3.497)}
\gppoint{gp mark 9}{(7.829,3.615)}
\gppoint{gp mark 9}{(6.902,5.092)}
\gpcolor{color=gp lt color border}
\node[gp node right] at (6.260,4.784) {GMM-EM};
\gpcolor{color=gp lt color 1}
\gpsetlinetype{gp lt plot 1}
\draw[gp path] (6.444,4.784)--(7.360,4.784);
\draw[gp path] (1.012,0.985)--(2.148,1.545)--(3.284,2.104)--(4.421,2.574)--(5.557,3.156)%
  --(6.693,3.643)--(7.829,3.910);
\gppoint{gp mark 10}{(1.012,0.985)}
\gppoint{gp mark 10}{(2.148,1.545)}
\gppoint{gp mark 10}{(3.284,2.104)}
\gppoint{gp mark 10}{(4.421,2.574)}
\gppoint{gp mark 10}{(5.557,3.156)}
\gppoint{gp mark 10}{(6.693,3.643)}
\gppoint{gp mark 10}{(7.829,3.910)}
\gppoint{gp mark 10}{(6.902,4.784)}
\gpcolor{color=gp lt color border}
\gpsetlinetype{gp lt border}
\draw[gp path] (1.012,4.133)--(1.012,0.985)--(7.829,0.985)--(7.829,4.133)--cycle;
%% coordinates of the plot area
\gpdefrectangularnode{gp plot 1}{\pgfpoint{1.012cm}{0.985cm}}{\pgfpoint{7.829cm}{4.133cm}}
\end{tikzpicture}
%% gnuplot variables

		\vspace{-10pt}
		\caption{The speedup of FlashMatrix with multithreading both in memory
		and on SSDs.}
		\label{fig:speedup}
	\end{center}
\end{figure}

The in-memory execution of FlashMatrix achieves almost linear speedup in all
applications while the out-of-core execution of some of the applications only
starts to flatten out after 32 threads (Figure \ref{fig:speedup}). Owing to
operation fusion in CPU cache, FlashMatrix significantly reduces data movement
between CPU and main memory. As such, memory bandwidth is no longer
the bottleneck and the computation speeds up linearly with more CPU cores
when the applications run in memory. The SSDs deliver about 10GB/s I/O throughput
and become the bottleneck in the applications with lower computation overhead.
Applications such as GMM, which have much higher computation complexity, still
speed up almost linearly even when running on SSDs. The performance results in
Figure \ref{fig:fmR} and Figure
\ref{fig:speedup} indicate that the applications executed in FlashMatrix can
achieve performance comparable to or even outperform parallel C or FORTRAN
implementations.

\begin{figure}
	\begin{center}
		\footnotesize
		\vspace{-15pt}
		\begin{tikzpicture}[gnuplot]
%% generated with GNUPLOT 4.6p4 (Lua 5.1; terminal rev. 99, script rev. 100)
%% Fri 08 Apr 2016 08:51:15 PM EDT
\path (0.000,0.000) rectangle (8.382,4.572);
\gpcolor{color=gp lt color border}
\gpsetlinetype{gp lt border}
\gpsetlinewidth{1.00}
\draw[gp path] (1.504,0.616)--(1.684,0.616);
\draw[gp path] (7.829,0.616)--(7.649,0.616);
\node[gp node right] at (1.320,0.616) { 0};
\draw[gp path] (1.504,1.282)--(1.684,1.282);
\draw[gp path] (7.829,1.282)--(7.649,1.282);
\node[gp node right] at (1.320,1.282) { 50};
\draw[gp path] (1.504,1.948)--(1.684,1.948);
\draw[gp path] (7.829,1.948)--(7.649,1.948);
\node[gp node right] at (1.320,1.948) { 100};
\draw[gp path] (1.504,2.613)--(1.684,2.613);
\draw[gp path] (7.829,2.613)--(7.649,2.613);
\node[gp node right] at (1.320,2.613) { 150};
\draw[gp path] (1.504,3.279)--(1.684,3.279);
\draw[gp path] (7.829,3.279)--(7.649,3.279);
\node[gp node right] at (1.320,3.279) { 200};
\draw[gp path] (3.085,0.616)--(3.085,0.796);
\draw[gp path] (3.085,3.279)--(3.085,3.099);
\node[gp node center] at (3.085,0.308) {64M};
\draw[gp path] (4.667,0.616)--(4.667,0.796);
\draw[gp path] (4.667,3.279)--(4.667,3.099);
\node[gp node center] at (4.667,0.308) {1B};
\draw[gp path] (6.248,0.616)--(6.248,0.796);
\draw[gp path] (6.248,3.279)--(6.248,3.099);
\node[gp node center] at (6.248,0.308) {4B};
\draw[gp path] (1.504,3.279)--(1.504,0.616)--(7.829,0.616)--(7.829,3.279)--cycle;
\node[gp node center,rotate=-270] at (0.246,1.947) {Runtime (s)};
\node[gp node right] at (3.382,4.238) {FM-IM};
\def\gpfillpath{(3.566,4.161)--(4.482,4.161)--(4.482,4.315)--(3.566,4.315)--cycle}
\gpfill{color=gpbgfillcolor} \gpfillpath;
\gpfill{color=gp lt color 0,gp pattern 0,pattern color=.} \gpfillpath;
\gpcolor{color=gp lt color 0}
\gpsetlinetype{gp lt plot 0}
\draw[gp path] (3.566,4.161)--(4.482,4.161)--(4.482,4.315)--(3.566,4.315)--cycle;
\def\gpfillpath{(2.690,0.616)--(2.954,0.616)--(2.954,0.632)--(2.690,0.632)--cycle}
\gpfill{color=gpbgfillcolor} \gpfillpath;
\gpfill{color=gp lt color 0,gp pattern 0,pattern color=.} \gpfillpath;
\draw[gp path] (2.690,0.616)--(2.690,0.631)--(2.953,0.631)--(2.953,0.616)--cycle;
\def\gpfillpath{(4.271,0.616)--(4.536,0.616)--(4.536,0.857)--(4.271,0.857)--cycle}
\gpfill{color=gpbgfillcolor} \gpfillpath;
\gpfill{color=gp lt color 0,gp pattern 0,pattern color=.} \gpfillpath;
\draw[gp path] (4.271,0.616)--(4.271,0.856)--(4.535,0.856)--(4.535,0.616)--cycle;
\gpcolor{color=gp lt color border}
\node[gp node right] at (3.382,3.930) {FM-EM};
\def\gpfillpath{(3.566,3.853)--(4.482,3.853)--(4.482,4.007)--(3.566,4.007)--cycle}
\gpfill{color=gpbgfillcolor} \gpfillpath;
\gpfill{color=gp lt color 1,gp pattern 1,pattern color=.} \gpfillpath;
\gpcolor{color=gp lt color 1}
\gpsetlinetype{gp lt plot 1}
\draw[gp path] (3.566,3.853)--(4.482,3.853)--(4.482,4.007)--(3.566,4.007)--cycle;
\def\gpfillpath{(2.953,0.616)--(3.218,0.616)--(3.218,0.641)--(2.953,0.641)--cycle}
\gpfill{color=gpbgfillcolor} \gpfillpath;
\gpfill{color=gp lt color 1,gp pattern 1,pattern color=.} \gpfillpath;
\draw[gp path] (2.953,0.616)--(2.953,0.640)--(3.217,0.640)--(3.217,0.616)--cycle;
\def\gpfillpath{(4.535,0.616)--(4.799,0.616)--(4.799,0.928)--(4.535,0.928)--cycle}
\gpfill{color=gpbgfillcolor} \gpfillpath;
\gpfill{color=gp lt color 1,gp pattern 1,pattern color=.} \gpfillpath;
\draw[gp path] (4.535,0.616)--(4.535,0.927)--(4.798,0.927)--(4.798,0.616)--cycle;
\def\gpfillpath{(6.116,0.616)--(6.381,0.616)--(6.381,1.891)--(6.116,1.891)--cycle}
\gpfill{color=gpbgfillcolor} \gpfillpath;
\gpfill{color=gp lt color 1,gp pattern 1,pattern color=.} \gpfillpath;
\draw[gp path] (6.116,0.616)--(6.116,1.890)--(6.380,1.890)--(6.380,0.616)--cycle;
\gpcolor{color=gp lt color border}
\node[gp node right] at (5.586,4.238) {MLlib};
\def\gpfillpath{(5.770,4.161)--(6.686,4.161)--(6.686,4.315)--(5.770,4.315)--cycle}
\gpfill{color=gpbgfillcolor} \gpfillpath;
\gpfill{color=gp lt color 2,gp pattern 2,pattern color=.} \gpfillpath;
\gpcolor{color=gp lt color 2}
\gpsetlinetype{gp lt plot 2}
\draw[gp path] (5.770,4.161)--(6.686,4.161)--(6.686,4.315)--(5.770,4.315)--cycle;
\def\gpfillpath{(3.217,0.616)--(3.482,0.616)--(3.482,0.677)--(3.217,0.677)--cycle}
\gpfill{color=gpbgfillcolor} \gpfillpath;
\gpfill{color=gp lt color 2,gp pattern 2,pattern color=.} \gpfillpath;
\draw[gp path] (3.217,0.616)--(3.217,0.676)--(3.481,0.676)--(3.481,0.616)--cycle;
\def\gpfillpath{(4.798,0.616)--(5.063,0.616)--(5.063,2.601)--(4.798,2.601)--cycle}
\gpfill{color=gpbgfillcolor} \gpfillpath;
\gpfill{color=gp lt color 2,gp pattern 2,pattern color=.} \gpfillpath;
\draw[gp path] (4.798,0.616)--(4.798,2.600)--(5.062,2.600)--(5.062,0.616)--cycle;
\gpcolor{color=gp lt color border}
\node[gp node right] at (5.586,3.930) {R};
\def\gpfillpath{(5.770,3.853)--(6.686,3.853)--(6.686,4.007)--(5.770,4.007)--cycle}
\gpfill{color=gpbgfillcolor} \gpfillpath;
\gpfill{color=gp lt color 3,gp pattern 3,pattern color=.} \gpfillpath;
\gpcolor{color=gp lt color 3}
\gpsetlinetype{gp lt plot 3}
\draw[gp path] (5.770,3.853)--(6.686,3.853)--(6.686,4.007)--(5.770,4.007)--cycle;
\def\gpfillpath{(3.481,0.616)--(3.745,0.616)--(3.745,3.014)--(3.481,3.014)--cycle}
\gpfill{color=gpbgfillcolor} \gpfillpath;
\gpfill{color=gp lt color 3,gp pattern 3,pattern color=.} \gpfillpath;
\draw[gp path] (3.481,0.616)--(3.481,3.013)--(3.744,3.013)--(3.744,0.616)--cycle;
\gpcolor{color=gp lt color border}
\gpsetlinetype{gp lt border}
\draw[gp path] (1.504,3.279)--(1.504,0.616)--(7.829,0.616)--(7.829,3.279)--cycle;
%% coordinates of the plot area
\gpdefrectangularnode{gp plot 1}{\pgfpoint{1.504cm}{0.616cm}}{\pgfpoint{7.829cm}{3.279cm}}
\end{tikzpicture}
%% gnuplot variables

		\vspace{-10pt}
		\caption{The scalability and performance of k-means in different
		frameworks.}
		\label{fig:scale}
	\end{center}
\end{figure}

\subsection{Performance of FlashMatrix in memory and on SSDs}

We further measure the in-memory and external-memory performance of the FlashMatrix
implementations of the applications thoroughly with different datasets and
different parameters. We run the first three applications on random matrices
with 64 million rows and vary the number of columns from 8 to 512. We run k-means
and GMM on the matrix constructed from 32 eigenvectors and vary the number of
clusters from 2 to 64.

\begin{figure}
	\begin{center}
		\footnotesize
		\vspace{-15pt}
		\begin{tikzpicture}[gnuplot]
%% generated with GNUPLOT 4.6p4 (Lua 5.1; terminal rev. 99, script rev. 100)
%% Sun 10 Apr 2016 09:18:38 PM EDT
\path (0.000,0.000) rectangle (8.382,4.572);
\gpcolor{color=gp lt color border}
\gpsetlinetype{gp lt border}
\gpsetlinewidth{1.00}
\draw[gp path] (1.196,0.985)--(1.376,0.985);
\draw[gp path] (7.829,0.985)--(7.649,0.985);
\node[gp node right] at (1.012,0.985) { 0};
\draw[gp path] (1.196,1.505)--(1.376,1.505);
\draw[gp path] (7.829,1.505)--(7.649,1.505);
\node[gp node right] at (1.012,1.505) { 0.2};
\draw[gp path] (1.196,2.026)--(1.376,2.026);
\draw[gp path] (7.829,2.026)--(7.649,2.026);
\node[gp node right] at (1.012,2.026) { 0.4};
\draw[gp path] (1.196,2.546)--(1.376,2.546);
\draw[gp path] (7.829,2.546)--(7.649,2.546);
\node[gp node right] at (1.012,2.546) { 0.6};
\draw[gp path] (1.196,3.067)--(1.376,3.067);
\draw[gp path] (7.829,3.067)--(7.649,3.067);
\node[gp node right] at (1.012,3.067) { 0.8};
\draw[gp path] (1.196,3.587)--(1.376,3.587);
\draw[gp path] (7.829,3.587)--(7.649,3.587);
\node[gp node right] at (1.012,3.587) { 1};
\draw[gp path] (1.196,0.985)--(1.196,1.165);
\draw[gp path] (1.196,3.587)--(1.196,3.407);
\node[gp node center] at (1.196,0.677) {8};
\draw[gp path] (3.407,0.985)--(3.407,1.165);
\draw[gp path] (3.407,3.587)--(3.407,3.407);
\node[gp node center] at (3.407,0.677) {32};
\draw[gp path] (5.618,0.985)--(5.618,1.165);
\draw[gp path] (5.618,3.587)--(5.618,3.407);
\node[gp node center] at (5.618,0.677) {128};
\draw[gp path] (7.829,0.985)--(7.829,1.165);
\draw[gp path] (7.829,3.587)--(7.829,3.407);
\node[gp node center] at (7.829,0.677) {512};
\draw[gp path] (1.196,3.587)--(1.196,0.985)--(7.829,0.985)--(7.829,3.587)--cycle;
\node[gp node center] at (4.512,0.215) {The number of columns};
\node[gp node right] at (1.942,4.238) {summary};
\gpcolor{color=gp lt color 0}
\gpsetlinetype{gp lt plot 0}
\draw[gp path] (2.126,4.238)--(3.042,4.238);
\draw[gp path] (1.196,1.662)--(3.407,1.844)--(5.618,2.364)--(7.829,2.598);
\gpsetpointsize{4.00}
\gppoint{gp mark 1}{(1.196,1.662)}
\gppoint{gp mark 1}{(3.407,1.844)}
\gppoint{gp mark 1}{(5.618,2.364)}
\gppoint{gp mark 1}{(7.829,2.598)}
\gppoint{gp mark 1}{(2.584,4.238)}
\gpcolor{color=gp lt color border}
\node[gp node right] at (4.514,4.238) {cor};
\gpcolor{color=gp lt color 1}
\gpsetlinetype{gp lt plot 1}
\draw[gp path] (4.698,4.238)--(5.614,4.238);
\draw[gp path] (1.196,1.766)--(3.407,2.520)--(5.618,2.989)--(7.829,2.989);
\gppoint{gp mark 2}{(1.196,1.766)}
\gppoint{gp mark 2}{(3.407,2.520)}
\gppoint{gp mark 2}{(5.618,2.989)}
\gppoint{gp mark 2}{(7.829,2.989)}
\gppoint{gp mark 2}{(5.156,4.238)}
\gpcolor{color=gp lt color border}
\node[gp node right] at (7.086,4.238) {SVD};
\gpcolor{color=gp lt color 2}
\gpsetlinetype{gp lt plot 2}
\draw[gp path] (7.270,4.238)--(8.186,4.238);
\draw[gp path] (1.196,1.531)--(3.407,2.390)--(5.618,3.561)--(7.829,3.535);
\gppoint{gp mark 3}{(1.196,1.531)}
\gppoint{gp mark 3}{(3.407,2.390)}
\gppoint{gp mark 3}{(5.618,3.561)}
\gppoint{gp mark 3}{(7.829,3.535)}
\gppoint{gp mark 3}{(7.728,4.238)}
\gpcolor{color=gp lt color border}
\gpsetlinetype{gp lt border}
\draw[gp path] (1.196,3.587)--(1.196,0.985)--(7.829,0.985)--(7.829,3.587)--cycle;
%% coordinates of the plot area
\gpdefrectangularnode{gp plot 1}{\pgfpoint{1.196cm}{0.985cm}}{\pgfpoint{7.829cm}{3.587cm}}
\end{tikzpicture}
%% gnuplot variables

		\vspace{-10pt}
		\caption{The relative external-memory performance of FlashMatrix for
			statistics computation on datasets with 64 million data points
			and the number of attributes varying from 8 to 512, normalized by
		its in-memory performance.}
		\label{perf:stat}
	\end{center}
\end{figure}

As the number of attributes in the datasets or the number of clusters increases,
the performance gap between in-memory and external-memory execution
narrows and eventually the external-memory performance gets almost 100\%
of in-memory performance (Figure \ref{perf:stat} and \ref{perf:clust}).
This observation conforms with the computation and I/O complexity of
the applications in Table \ref{tbl:algs}. When the number of attributes
in the dataset gets larger, computation in correlation and SVD grows more
rapidly than I/O and eventually CPU becomes
the bottleneck. Similarly, the computation of k-means and GMM increases
more rapidly than I/O and get dominated by their CPU computation as the number
of clusters gets larger. Given the I/O throughput of 10 GB/s, the applications
do not require many attributes or clusters to have their external-memory
performance close to their in-memory performance.

\begin{figure}
	\begin{center}
		\footnotesize
		\vspace{-15pt}
		\begin{tikzpicture}[gnuplot]
%% generated with GNUPLOT 4.6p4 (Lua 5.1; terminal rev. 99, script rev. 100)
%% Wed 30 Mar 2016 03:56:24 PM EDT
\path (0.000,0.000) rectangle (8.382,4.572);
\gpcolor{color=gp lt color border}
\gpsetlinetype{gp lt border}
\gpsetlinewidth{1.00}
\draw[gp path] (1.196,0.985)--(1.376,0.985);
\draw[gp path] (7.829,0.985)--(7.649,0.985);
\node[gp node right] at (1.012,0.985) { 0};
\draw[gp path] (1.196,1.445)--(1.376,1.445);
\draw[gp path] (7.829,1.445)--(7.649,1.445);
\node[gp node right] at (1.012,1.445) { 0.5};
\draw[gp path] (1.196,1.904)--(1.376,1.904);
\draw[gp path] (7.829,1.904)--(7.649,1.904);
\node[gp node right] at (1.012,1.904) { 1};
\draw[gp path] (1.196,2.364)--(1.376,2.364);
\draw[gp path] (7.829,2.364)--(7.649,2.364);
\node[gp node right] at (1.012,2.364) { 1.5};
\draw[gp path] (1.196,2.824)--(1.376,2.824);
\draw[gp path] (7.829,2.824)--(7.649,2.824);
\node[gp node right] at (1.012,2.824) { 2};
\draw[gp path] (1.196,3.284)--(1.376,3.284);
\draw[gp path] (7.829,3.284)--(7.649,3.284);
\node[gp node right] at (1.012,3.284) { 2.5};
\draw[gp path] (1.196,3.743)--(1.376,3.743);
\draw[gp path] (7.829,3.743)--(7.649,3.743);
\node[gp node right] at (1.012,3.743) { 3};
\draw[gp path] (1.196,4.203)--(1.376,4.203);
\draw[gp path] (7.829,4.203)--(7.649,4.203);
\node[gp node right] at (1.012,4.203) { 3.5};
\draw[gp path] (1.196,0.985)--(1.196,1.165);
\draw[gp path] (1.196,4.203)--(1.196,4.023);
\node[gp node center] at (1.196,0.677) {2};
\draw[gp path] (2.523,0.985)--(2.523,1.165);
\draw[gp path] (2.523,4.203)--(2.523,4.023);
\node[gp node center] at (2.523,0.677) {4};
\draw[gp path] (3.849,0.985)--(3.849,1.165);
\draw[gp path] (3.849,4.203)--(3.849,4.023);
\node[gp node center] at (3.849,0.677) {8};
\draw[gp path] (5.176,0.985)--(5.176,1.165);
\draw[gp path] (5.176,4.203)--(5.176,4.023);
\node[gp node center] at (5.176,0.677) {16};
\draw[gp path] (6.502,0.985)--(6.502,1.165);
\draw[gp path] (6.502,4.203)--(6.502,4.023);
\node[gp node center] at (6.502,0.677) {32};
\draw[gp path] (7.829,0.985)--(7.829,1.165);
\draw[gp path] (7.829,4.203)--(7.829,4.023);
\node[gp node center] at (7.829,0.677) {64};
\draw[gp path] (1.196,4.203)--(1.196,0.985)--(7.829,0.985)--(7.829,4.203)--cycle;
\node[gp node center] at (4.512,0.215) {The number of clusters};
\node[gp node right] at (6.361,3.869) {KMeans};
\gpcolor{color=gp lt color 0}
\gpsetlinetype{gp lt plot 0}
\draw[gp path] (6.545,3.869)--(7.461,3.869);
\draw[gp path] (1.196,3.964)--(2.523,3.247)--(3.849,2.566)--(5.176,2.153)--(6.502,2.061)%
  --(7.829,1.914);
\gpsetpointsize{4.00}
\gppoint{gp mark 1}{(1.196,3.964)}
\gppoint{gp mark 1}{(2.523,3.247)}
\gppoint{gp mark 1}{(3.849,2.566)}
\gppoint{gp mark 1}{(5.176,2.153)}
\gppoint{gp mark 1}{(6.502,2.061)}
\gppoint{gp mark 1}{(7.829,1.914)}
\gppoint{gp mark 1}{(7.003,3.869)}
\gpcolor{color=gp lt color border}
\node[gp node right] at (6.361,3.561) {GMM};
\gpcolor{color=gp lt color 1}
\gpsetlinetype{gp lt plot 1}
\draw[gp path] (6.545,3.561)--(7.461,3.561);
\draw[gp path] (1.196,2.953)--(2.523,2.309)--(3.849,2.125)--(5.176,2.116)--(6.502,2.015)%
  --(7.829,1.996);
\gppoint{gp mark 2}{(1.196,2.953)}
\gppoint{gp mark 2}{(2.523,2.309)}
\gppoint{gp mark 2}{(3.849,2.125)}
\gppoint{gp mark 2}{(5.176,2.116)}
\gppoint{gp mark 2}{(6.502,2.015)}
\gppoint{gp mark 2}{(7.829,1.996)}
\gppoint{gp mark 2}{(7.003,3.561)}
\gpcolor{color=gp lt color border}
\gpsetlinetype{gp lt border}
\draw[gp path] (1.196,4.203)--(1.196,0.985)--(7.829,0.985)--(7.829,4.203)--cycle;
%% coordinates of the plot area
\gpdefrectangularnode{gp plot 1}{\pgfpoint{1.196cm}{0.985cm}}{\pgfpoint{7.829cm}{4.203cm}}
\end{tikzpicture}
%% gnuplot variables

		\vspace{-10pt}
		\caption{The relative external-memory performance of FlashMatrix for
			clustering algorithms with different numbers of clusters, normalized
		by its in-memory performance.}
		\label{perf:clust}
	\end{center}
\end{figure}

\subsection{Effectiveness of optimizations}

In this section, we illustrate the effectiveness of our memory and CPU
optimizations in FlashMatrix. To reduce memory overhead, we focus on three
main optimizations: \textit{(i)} reusing memory chunks for new in-memory
matrices and I/O access to reduce large
memory allocation, \textit{(ii)} matrix operation fusion in main
memory to reduce data movement between SSDs and main memory, \textit{(iii)}
matrix operation fusion in CPU cache to reduce data movement between main
memory and CPU cache. To reduce computation overhead, we illustrate
the effectiveness of using VUDFs.

Each memory optimization has significant performance improvement on most of
the applications when they are executed on SSDs (Figure \ref{perf:opts}).
Due to the space limit, we only illustrate the effectiveness of
the memory optimizations on SSDs. Operation fusion in main memory achieves
the highest performance improvement in almost all applications, even in GMM,
which has the highest asymptotic computation complexity. Even though the SSDs
deliver 10GB/s I/O throughput, materializing every matrix operation separately
causes SSDs to be the main bottleneck in the system.
Fusing matrix operations in memory significantly reduces the burden on SSDs and
improves performance by a large factor. Operation fusion in the CPU cache also
has very positive performance impact on some applications even when
the applications run on SSDs. This suggests that with sufficient I/O optimizations,
many machine learning applications that run on fast SSDs can be bottlenecked by
the bandwidth of main memory, instead of I/O. Even though it is less noticeable,
reducing large memory allocation improves I/O performance and almost doubles
the performance for all applications.

\begin{figure}
	\begin{center}
		\footnotesize
		\vspace{-15pt}
		\begin{tikzpicture}[gnuplot]
%% generated with GNUPLOT 4.6p4 (Lua 5.1; terminal rev. 99, script rev. 100)
%% Mon 04 Apr 2016 09:54:30 PM EDT
\path (0.000,0.000) rectangle (8.382,4.572);
\gpcolor{color=gp lt color border}
\gpsetlinetype{gp lt border}
\gpsetlinewidth{1.00}
\draw[gp path] (1.320,0.616)--(1.500,0.616);
\draw[gp path] (7.829,0.616)--(7.649,0.616);
\node[gp node right] at (1.136,0.616) { 0};
\draw[gp path] (1.320,1.149)--(1.500,1.149);
\draw[gp path] (7.829,1.149)--(7.649,1.149);
\node[gp node right] at (1.136,1.149) { 5};
\draw[gp path] (1.320,1.681)--(1.500,1.681);
\draw[gp path] (7.829,1.681)--(7.649,1.681);
\node[gp node right] at (1.136,1.681) { 10};
\draw[gp path] (1.320,2.214)--(1.500,2.214);
\draw[gp path] (7.829,2.214)--(7.649,2.214);
\node[gp node right] at (1.136,2.214) { 15};
\draw[gp path] (1.320,2.746)--(1.500,2.746);
\draw[gp path] (7.829,2.746)--(7.649,2.746);
\node[gp node right] at (1.136,2.746) { 20};
\draw[gp path] (1.320,3.279)--(1.500,3.279);
\draw[gp path] (7.829,3.279)--(7.649,3.279);
\node[gp node right] at (1.136,3.279) { 25};
\draw[gp path] (2.405,0.616)--(2.405,0.796);
\draw[gp path] (2.405,3.279)--(2.405,3.099);
\node[gp node center] at (2.405,0.308) {summary};
\draw[gp path] (3.490,0.616)--(3.490,0.796);
\draw[gp path] (3.490,3.279)--(3.490,3.099);
\node[gp node center] at (3.490,0.308) {cor};
\draw[gp path] (4.575,0.616)--(4.575,0.796);
\draw[gp path] (4.575,3.279)--(4.575,3.099);
\node[gp node center] at (4.575,0.308) {SVD};
\draw[gp path] (5.659,0.616)--(5.659,0.796);
\draw[gp path] (5.659,3.279)--(5.659,3.099);
\node[gp node center] at (5.659,0.308) {KMeans};
\draw[gp path] (6.744,0.616)--(6.744,0.796);
\draw[gp path] (6.744,3.279)--(6.744,3.099);
\node[gp node center] at (6.744,0.308) {GMM};
\draw[gp path] (1.320,3.279)--(1.320,0.616)--(7.829,0.616)--(7.829,3.279)--cycle;
\node[gp node center,rotate=-270] at (0.246,1.947) {relative performance};
\node[gp node right] at (3.290,4.238) {Base};
\def\gpfillpath{(3.474,4.161)--(4.390,4.161)--(4.390,4.315)--(3.474,4.315)--cycle}
\gpfill{color=gpbgfillcolor} \gpfillpath;
\gpfill{color=gp lt color 0,gp pattern 0,pattern color=.} \gpfillpath;
\gpcolor{color=gp lt color 0}
\gpsetlinetype{gp lt plot 0}
\draw[gp path] (3.474,4.161)--(4.390,4.161)--(4.390,4.315)--(3.474,4.315)--cycle;
\def\gpfillpath{(2.134,0.616)--(2.315,0.616)--(2.315,0.724)--(2.134,0.724)--cycle}
\gpfill{color=gpbgfillcolor} \gpfillpath;
\gpfill{color=gp lt color 0,gp pattern 0,pattern color=.} \gpfillpath;
\draw[gp path] (2.134,0.616)--(2.134,0.723)--(2.314,0.723)--(2.314,0.616)--cycle;
\def\gpfillpath{(3.218,0.616)--(3.400,0.616)--(3.400,0.724)--(3.218,0.724)--cycle}
\gpfill{color=gpbgfillcolor} \gpfillpath;
\gpfill{color=gp lt color 0,gp pattern 0,pattern color=.} \gpfillpath;
\draw[gp path] (3.218,0.616)--(3.218,0.723)--(3.399,0.723)--(3.399,0.616)--cycle;
\def\gpfillpath{(4.303,0.616)--(4.485,0.616)--(4.485,0.724)--(4.303,0.724)--cycle}
\gpfill{color=gpbgfillcolor} \gpfillpath;
\gpfill{color=gp lt color 0,gp pattern 0,pattern color=.} \gpfillpath;
\draw[gp path] (4.303,0.616)--(4.303,0.723)--(4.484,0.723)--(4.484,0.616)--cycle;
\def\gpfillpath{(5.388,0.616)--(5.570,0.616)--(5.570,0.724)--(5.388,0.724)--cycle}
\gpfill{color=gpbgfillcolor} \gpfillpath;
\gpfill{color=gp lt color 0,gp pattern 0,pattern color=.} \gpfillpath;
\draw[gp path] (5.388,0.616)--(5.388,0.723)--(5.569,0.723)--(5.569,0.616)--cycle;
\def\gpfillpath{(6.473,0.616)--(6.655,0.616)--(6.655,0.724)--(6.473,0.724)--cycle}
\gpfill{color=gpbgfillcolor} \gpfillpath;
\gpfill{color=gp lt color 0,gp pattern 0,pattern color=.} \gpfillpath;
\draw[gp path] (6.473,0.616)--(6.473,0.723)--(6.654,0.723)--(6.654,0.616)--cycle;
\gpcolor{color=gp lt color border}
\node[gp node right] at (3.290,3.930) {Mem-alloc};
\def\gpfillpath{(3.474,3.853)--(4.390,3.853)--(4.390,4.007)--(3.474,4.007)--cycle}
\gpfill{color=gpbgfillcolor} \gpfillpath;
\gpfill{color=gp lt color 1,gp pattern 1,pattern color=.} \gpfillpath;
\gpcolor{color=gp lt color 1}
\gpsetlinetype{gp lt plot 1}
\draw[gp path] (3.474,3.853)--(4.390,3.853)--(4.390,4.007)--(3.474,4.007)--cycle;
\def\gpfillpath{(2.314,0.616)--(2.496,0.616)--(2.496,0.844)--(2.314,0.844)--cycle}
\gpfill{color=gpbgfillcolor} \gpfillpath;
\gpfill{color=gp lt color 1,gp pattern 1,pattern color=.} \gpfillpath;
\draw[gp path] (2.314,0.616)--(2.314,0.843)--(2.495,0.843)--(2.495,0.616)--cycle;
\def\gpfillpath{(3.399,0.616)--(3.581,0.616)--(3.581,0.872)--(3.399,0.872)--cycle}
\gpfill{color=gpbgfillcolor} \gpfillpath;
\gpfill{color=gp lt color 1,gp pattern 1,pattern color=.} \gpfillpath;
\draw[gp path] (3.399,0.616)--(3.399,0.871)--(3.580,0.871)--(3.580,0.616)--cycle;
\def\gpfillpath{(4.484,0.616)--(4.666,0.616)--(4.666,0.861)--(4.484,0.861)--cycle}
\gpfill{color=gpbgfillcolor} \gpfillpath;
\gpfill{color=gp lt color 1,gp pattern 1,pattern color=.} \gpfillpath;
\draw[gp path] (4.484,0.616)--(4.484,0.860)--(4.665,0.860)--(4.665,0.616)--cycle;
\def\gpfillpath{(5.569,0.616)--(5.751,0.616)--(5.751,0.764)--(5.569,0.764)--cycle}
\gpfill{color=gpbgfillcolor} \gpfillpath;
\gpfill{color=gp lt color 1,gp pattern 1,pattern color=.} \gpfillpath;
\draw[gp path] (5.569,0.616)--(5.569,0.763)--(5.750,0.763)--(5.750,0.616)--cycle;
\def\gpfillpath{(6.654,0.616)--(6.836,0.616)--(6.836,0.908)--(6.654,0.908)--cycle}
\gpfill{color=gpbgfillcolor} \gpfillpath;
\gpfill{color=gp lt color 1,gp pattern 1,pattern color=.} \gpfillpath;
\draw[gp path] (6.654,0.616)--(6.654,0.907)--(6.835,0.907)--(6.835,0.616)--cycle;
\gpcolor{color=gp lt color border}
\node[gp node right] at (6.414,4.238) {Mem-fuse};
\def\gpfillpath{(6.598,4.161)--(7.514,4.161)--(7.514,4.315)--(6.598,4.315)--cycle}
\gpfill{color=gpbgfillcolor} \gpfillpath;
\gpfill{color=gp lt color 2,gp pattern 2,pattern color=.} \gpfillpath;
\gpcolor{color=gp lt color 2}
\gpsetlinetype{gp lt plot 2}
\draw[gp path] (6.598,4.161)--(7.514,4.161)--(7.514,4.315)--(6.598,4.315)--cycle;
\def\gpfillpath{(2.495,0.616)--(2.677,0.616)--(2.677,2.554)--(2.495,2.554)--cycle}
\gpfill{color=gpbgfillcolor} \gpfillpath;
\gpfill{color=gp lt color 2,gp pattern 2,pattern color=.} \gpfillpath;
\draw[gp path] (2.495,0.616)--(2.495,2.553)--(2.676,2.553)--(2.676,0.616)--cycle;
\def\gpfillpath{(3.580,0.616)--(3.762,0.616)--(3.762,1.611)--(3.580,1.611)--cycle}
\gpfill{color=gpbgfillcolor} \gpfillpath;
\gpfill{color=gp lt color 2,gp pattern 2,pattern color=.} \gpfillpath;
\draw[gp path] (3.580,0.616)--(3.580,1.610)--(3.761,1.610)--(3.761,0.616)--cycle;
\def\gpfillpath{(4.665,0.616)--(4.847,0.616)--(4.847,0.852)--(4.665,0.852)--cycle}
\gpfill{color=gpbgfillcolor} \gpfillpath;
\gpfill{color=gp lt color 2,gp pattern 2,pattern color=.} \gpfillpath;
\draw[gp path] (4.665,0.616)--(4.665,0.851)--(4.846,0.851)--(4.846,0.616)--cycle;
\def\gpfillpath{(5.750,0.616)--(5.932,0.616)--(5.932,1.437)--(5.750,1.437)--cycle}
\gpfill{color=gpbgfillcolor} \gpfillpath;
\gpfill{color=gp lt color 2,gp pattern 2,pattern color=.} \gpfillpath;
\draw[gp path] (5.750,0.616)--(5.750,1.436)--(5.931,1.436)--(5.931,0.616)--cycle;
\def\gpfillpath{(6.835,0.616)--(7.016,0.616)--(7.016,2.204)--(6.835,2.204)--cycle}
\gpfill{color=gpbgfillcolor} \gpfillpath;
\gpfill{color=gp lt color 2,gp pattern 2,pattern color=.} \gpfillpath;
\draw[gp path] (6.835,0.616)--(6.835,2.203)--(7.015,2.203)--(7.015,0.616)--cycle;
\gpcolor{color=gp lt color border}
\node[gp node right] at (6.414,3.930) {Cache-fuse};
\def\gpfillpath{(6.598,3.853)--(7.514,3.853)--(7.514,4.007)--(6.598,4.007)--cycle}
\gpfill{color=gpbgfillcolor} \gpfillpath;
\gpfill{color=gp lt color 3,gp pattern 3,pattern color=.} \gpfillpath;
\gpcolor{color=gp lt color 3}
\gpsetlinetype{gp lt plot 3}
\draw[gp path] (6.598,3.853)--(7.514,3.853)--(7.514,4.007)--(6.598,4.007)--cycle;
\def\gpfillpath{(2.676,0.616)--(2.858,0.616)--(2.858,3.230)--(2.676,3.230)--cycle}
\gpfill{color=gpbgfillcolor} \gpfillpath;
\gpfill{color=gp lt color 3,gp pattern 3,pattern color=.} \gpfillpath;
\draw[gp path] (2.676,0.616)--(2.676,3.229)--(2.857,3.229)--(2.857,0.616)--cycle;
\def\gpfillpath{(3.761,0.616)--(3.943,0.616)--(3.943,1.842)--(3.761,1.842)--cycle}
\gpfill{color=gpbgfillcolor} \gpfillpath;
\gpfill{color=gp lt color 3,gp pattern 3,pattern color=.} \gpfillpath;
\draw[gp path] (3.761,0.616)--(3.761,1.841)--(3.942,1.841)--(3.942,0.616)--cycle;
\def\gpfillpath{(4.846,0.616)--(5.028,0.616)--(5.028,0.876)--(4.846,0.876)--cycle}
\gpfill{color=gpbgfillcolor} \gpfillpath;
\gpfill{color=gp lt color 3,gp pattern 3,pattern color=.} \gpfillpath;
\draw[gp path] (4.846,0.616)--(4.846,0.875)--(5.027,0.875)--(5.027,0.616)--cycle;
\def\gpfillpath{(5.931,0.616)--(6.112,0.616)--(6.112,1.499)--(5.931,1.499)--cycle}
\gpfill{color=gpbgfillcolor} \gpfillpath;
\gpfill{color=gp lt color 3,gp pattern 3,pattern color=.} \gpfillpath;
\draw[gp path] (5.931,0.616)--(5.931,1.498)--(6.111,1.498)--(6.111,0.616)--cycle;
\def\gpfillpath{(7.015,0.616)--(7.197,0.616)--(7.197,2.807)--(7.015,2.807)--cycle}
\gpfill{color=gpbgfillcolor} \gpfillpath;
\gpfill{color=gp lt color 3,gp pattern 3,pattern color=.} \gpfillpath;
\draw[gp path] (7.015,0.616)--(7.015,2.806)--(7.196,2.806)--(7.196,0.616)--cycle;
\gpcolor{color=gp lt color border}
\gpsetlinetype{gp lt border}
\draw[gp path] (1.320,3.279)--(1.320,0.616)--(7.829,0.616)--(7.829,3.279)--cycle;
%% coordinates of the plot area
\gpdefrectangularnode{gp plot 1}{\pgfpoint{1.320cm}{0.616cm}}{\pgfpoint{7.829cm}{3.279cm}}
\end{tikzpicture}
%% gnuplot variables

		\vspace{-10pt}
		\caption{The effectiveness of memory optimizations on different
			applications when they are executed on SSDs. The three memory
		optimizations are applied to FlashMatrix incrementally.}
		\label{perf:opts}
	\end{center}
\end{figure}

Using VUDFs improves the performance of the applications that
rely on GenOps for computation (Figure \ref{perf:opts_CPU}). To remove memory
as the bottleneck from the system, the base implementations deploy all memory
optimizations and invoke user-defined functions on individual elements in
matrices. The FlashMatrix implementations of computing statistical summary
and k-means solely rely on GenOps. Therefore, their performance is almost doubled
when using VUDFs. The main computation in correlation and GMM is matrix
multiplication, but they still rely on GenOps for the remaining computation.
As such, using VUDFs help their performance. SVD solely uses matrix
multiplication, so switching VUDFs has no performance impact.

\begin{figure}
	\begin{center}
		\footnotesize
		\vspace{-15pt}
		\begin{tikzpicture}[gnuplot]
%% generated with GNUPLOT 4.6p4 (Lua 5.1; terminal rev. 99, script rev. 100)
%% Sat 09 Apr 2016 04:07:24 PM EDT
\path (0.000,0.000) rectangle (8.382,4.572);
\gpcolor{color=gp lt color border}
\gpsetlinetype{gp lt border}
\gpsetlinewidth{1.00}
\draw[gp path] (1.504,0.616)--(1.684,0.616);
\draw[gp path] (7.829,0.616)--(7.649,0.616);
\node[gp node right] at (1.320,0.616) { 0};
\draw[gp path] (1.504,1.111)--(1.684,1.111);
\draw[gp path] (7.829,1.111)--(7.649,1.111);
\node[gp node right] at (1.320,1.111) { 0.5};
\draw[gp path] (1.504,1.606)--(1.684,1.606);
\draw[gp path] (7.829,1.606)--(7.649,1.606);
\node[gp node right] at (1.320,1.606) { 1};
\draw[gp path] (1.504,2.102)--(1.684,2.102);
\draw[gp path] (7.829,2.102)--(7.649,2.102);
\node[gp node right] at (1.320,2.102) { 1.5};
\draw[gp path] (1.504,2.597)--(1.684,2.597);
\draw[gp path] (7.829,2.597)--(7.649,2.597);
\node[gp node right] at (1.320,2.597) { 2};
\draw[gp path] (1.504,3.092)--(1.684,3.092);
\draw[gp path] (7.829,3.092)--(7.649,3.092);
\node[gp node right] at (1.320,3.092) { 2.5};
\draw[gp path] (1.504,3.587)--(1.684,3.587);
\draw[gp path] (7.829,3.587)--(7.649,3.587);
\node[gp node right] at (1.320,3.587) { 3};
\draw[gp path] (2.558,0.616)--(2.558,0.796);
\draw[gp path] (2.558,3.587)--(2.558,3.407);
\node[gp node center] at (2.558,0.308) {summary};
\draw[gp path] (3.612,0.616)--(3.612,0.796);
\draw[gp path] (3.612,3.587)--(3.612,3.407);
\node[gp node center] at (3.612,0.308) {cor};
\draw[gp path] (4.667,0.616)--(4.667,0.796);
\draw[gp path] (4.667,3.587)--(4.667,3.407);
\node[gp node center] at (4.667,0.308) {SVD};
\draw[gp path] (5.721,0.616)--(5.721,0.796);
\draw[gp path] (5.721,3.587)--(5.721,3.407);
\node[gp node center] at (5.721,0.308) {KMeans};
\draw[gp path] (6.775,0.616)--(6.775,0.796);
\draw[gp path] (6.775,3.587)--(6.775,3.407);
\node[gp node center] at (6.775,0.308) {GMM};
\draw[gp path] (1.504,3.587)--(1.504,0.616)--(7.829,0.616)--(7.829,3.587)--cycle;
\node[gp node center,rotate=-270] at (0.246,2.101) {relative performance};
\node[gp node right] at (3.382,4.238) {Base};
\def\gpfillpath{(3.566,4.161)--(4.482,4.161)--(4.482,4.315)--(3.566,4.315)--cycle}
\gpfill{color=gpbgfillcolor} \gpfillpath;
\gpfill{color=gp lt color 0,gp pattern 0,pattern color=.} \gpfillpath;
\gpcolor{color=gp lt color 0}
\gpsetlinetype{gp lt plot 0}
\draw[gp path] (3.566,4.161)--(4.482,4.161)--(4.482,4.315)--(3.566,4.315)--cycle;
\def\gpfillpath{(2.426,0.616)--(2.691,0.616)--(2.691,1.607)--(2.426,1.607)--cycle}
\gpfill{color=gpbgfillcolor} \gpfillpath;
\gpfill{color=gp lt color 0,gp pattern 0,pattern color=.} \gpfillpath;
\draw[gp path] (2.426,0.616)--(2.426,1.606)--(2.690,1.606)--(2.690,0.616)--cycle;
\def\gpfillpath{(3.481,0.616)--(3.745,0.616)--(3.745,1.607)--(3.481,1.607)--cycle}
\gpfill{color=gpbgfillcolor} \gpfillpath;
\gpfill{color=gp lt color 0,gp pattern 0,pattern color=.} \gpfillpath;
\draw[gp path] (3.481,0.616)--(3.481,1.606)--(3.744,1.606)--(3.744,0.616)--cycle;
\def\gpfillpath{(4.535,0.616)--(4.799,0.616)--(4.799,1.607)--(4.535,1.607)--cycle}
\gpfill{color=gpbgfillcolor} \gpfillpath;
\gpfill{color=gp lt color 0,gp pattern 0,pattern color=.} \gpfillpath;
\draw[gp path] (4.535,0.616)--(4.535,1.606)--(4.798,1.606)--(4.798,0.616)--cycle;
\def\gpfillpath{(5.589,0.616)--(5.853,0.616)--(5.853,1.607)--(5.589,1.607)--cycle}
\gpfill{color=gpbgfillcolor} \gpfillpath;
\gpfill{color=gp lt color 0,gp pattern 0,pattern color=.} \gpfillpath;
\draw[gp path] (5.589,0.616)--(5.589,1.606)--(5.852,1.606)--(5.852,0.616)--cycle;
\def\gpfillpath{(6.643,0.616)--(6.908,0.616)--(6.908,1.607)--(6.643,1.607)--cycle}
\gpfill{color=gpbgfillcolor} \gpfillpath;
\gpfill{color=gp lt color 0,gp pattern 0,pattern color=.} \gpfillpath;
\draw[gp path] (6.643,0.616)--(6.643,1.606)--(6.907,1.606)--(6.907,0.616)--cycle;
\gpcolor{color=gp lt color border}
\node[gp node right] at (5.402,4.238) {VUDF};
\def\gpfillpath{(5.586,4.161)--(6.502,4.161)--(6.502,4.315)--(5.586,4.315)--cycle}
\gpfill{color=gpbgfillcolor} \gpfillpath;
\gpfill{color=gp lt color 1,gp pattern 1,pattern color=.} \gpfillpath;
\gpcolor{color=gp lt color 1}
\gpsetlinetype{gp lt plot 1}
\draw[gp path] (5.586,4.161)--(6.502,4.161)--(6.502,4.315)--(5.586,4.315)--cycle;
\def\gpfillpath{(2.690,0.616)--(2.954,0.616)--(2.954,2.766)--(2.690,2.766)--cycle}
\gpfill{color=gpbgfillcolor} \gpfillpath;
\gpfill{color=gp lt color 1,gp pattern 1,pattern color=.} \gpfillpath;
\draw[gp path] (2.690,0.616)--(2.690,2.765)--(2.953,2.765)--(2.953,0.616)--cycle;
\def\gpfillpath{(3.744,0.616)--(4.009,0.616)--(4.009,1.885)--(3.744,1.885)--cycle}
\gpfill{color=gpbgfillcolor} \gpfillpath;
\gpfill{color=gp lt color 1,gp pattern 1,pattern color=.} \gpfillpath;
\draw[gp path] (3.744,0.616)--(3.744,1.884)--(4.008,1.884)--(4.008,0.616)--cycle;
\def\gpfillpath{(4.798,0.616)--(5.063,0.616)--(5.063,1.607)--(4.798,1.607)--cycle}
\gpfill{color=gpbgfillcolor} \gpfillpath;
\gpfill{color=gp lt color 1,gp pattern 1,pattern color=.} \gpfillpath;
\draw[gp path] (4.798,0.616)--(4.798,1.606)--(5.062,1.606)--(5.062,0.616)--cycle;
\def\gpfillpath{(5.852,0.616)--(6.117,0.616)--(6.117,2.419)--(5.852,2.419)--cycle}
\gpfill{color=gpbgfillcolor} \gpfillpath;
\gpfill{color=gp lt color 1,gp pattern 1,pattern color=.} \gpfillpath;
\draw[gp path] (5.852,0.616)--(5.852,2.418)--(6.116,2.418)--(6.116,0.616)--cycle;
\def\gpfillpath{(6.907,0.616)--(7.171,0.616)--(7.171,1.736)--(6.907,1.736)--cycle}
\gpfill{color=gpbgfillcolor} \gpfillpath;
\gpfill{color=gp lt color 1,gp pattern 1,pattern color=.} \gpfillpath;
\draw[gp path] (6.907,0.616)--(6.907,1.735)--(7.170,1.735)--(7.170,0.616)--cycle;
\gpcolor{color=gp lt color border}
\gpsetlinetype{gp lt border}
\draw[gp path] (1.504,3.587)--(1.504,0.616)--(7.829,0.616)--(7.829,3.587)--cycle;
%% coordinates of the plot area
\gpdefrectangularnode{gp plot 1}{\pgfpoint{1.504cm}{0.616cm}}{\pgfpoint{7.829cm}{3.587cm}}
\end{tikzpicture}
%% gnuplot variables

		\vspace{-10pt}
		\caption{The effectiveness of using VUDFs on different applications
		when they are executed in memory.}
		\label{perf:opts_CPU}
	\end{center}
\end{figure}
