We present FlashMatrix, a matrix-oriented programming framework for general
data analysis. FlashMatrix scales to large datasets by utilizing commodity SSDs.
It provides a high-level functional programming interface for users to write
data analysis algorithms in R and
executes the R implementations in parallel and out of core automatically.
%For simplicity and generality, the core of FlashMatrix only implements
%a small number of generalized matrix operators (GenOps). It reimplements
%many matrix operations in R \textit{base} package with GenOps to provide
%a familiar programming environment to users. To improve performance,
%FlashMatrix uses vectorized element functions (VEleFuns) to reduce the
%overhead of function calls and fuses matrix operations to reduce data movement
%between CPU and SSDs.

We demonstrate that the matrix-oriented functional programming interface in
FlashMatrix achieves high performance and scalability for many machine learning and 
statistics algorithms.  R implementations executed in FlashMatrix outperforms
Spark MLlib on all inputs by a factor of 3 to 10, using the same shared memory hardware.
Furthermore, executing R programs with a single thread in FlashMatrix is more
efficient than the optimized C and Fortran implementations of native R packages.

%We implement multiple statistics and
%machine learning algorithms in R and compare their performance with Spark
%MLlib, a highly-optimized parallel machine learning library, on large datasets.
%The R implementations executed in FlashMatrix significantly outperforms
%the implementations in Spark MLlib. We further demonstrate that
%the R implementations running FlashMatrix with a single thread can outperform
%the C and FORTRAN implementations in the R framework. In addition, FlashMatrix
%also achieves linear speedup with multithreading in all of these algorithms.

Even though SSDs are still an order of magnitude slower than DRAM, the external-memory
execution of many data analysis algorithms in FlashMatrix can achieve performance
that approaches to their in-memory execution. We demonstrate that an I/O throughput
of 10 GB/s is able to saturate CPU for many algorithms, even in a large parallel
NUMA machine. 
%As such, the external-memory execution also benefits from many in-memory optimizations.

FlashMatrix simplifies the programming effort of writing
parallel and out-of-core implementations for large-scale data analysis. It
provides domain experts a familiar programming environment for implementing
their algorithms. It also significantly
increases productivity of writing an efficient implementation with performance
comparable to low-level programming languages. We believe FlashMatrix opens
a new opportunity for large-scale data analysis.
