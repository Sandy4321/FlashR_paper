We present FlashR, a matrix-oriented programming framework that executes
R-programmed machine learning algorithms in parallel and out-of-core
automatically. FlashR scales to large datasets by utilizing commodity SSDs.

%For simplicity and generality, the core of FlashR only implements
%a small number of generalized matrix operators (GenOps). It reimplements
%many matrix operations in R \textit{base} package with GenOps to provide
%a familiar programming environment to users. To improve performance,
%FlashR uses vectorized element functions (VEleFuns) to reduce the
%overhead of function calls and fuses matrix operations to reduce data movement
%between CPU and SSDs.

Although R is considered slow and unable to scale to large datasets,
we demonstrate that with sufficient system-level optimizations, FlashR powers
the R programming interface to achieve high performance and scalability
for developing many machine learning algorithms. R implementations executed in FlashR
outperform H$_2$O and Spark MLlib on all algorithms by a factor of $3-20$, using
the same shared memory hardware. FlashR scales to datasets with billions of
data points easily with negligible amounts of memory and completes all
algorithms within a reasonable amount of time.

Even though the current I/O technologies, such as solid-state drives (SSDs),
are an order of magnitude slower than DRAM, the external-memory execution
of many algorithms in FlashR achieves performance approaching their in-memory
execution. As the number of features and other factors, such as the number of
clusters in clustering algorithms, increase, we expect FlashR on SSDs to achieve
the same performance as
in memory. We demonstrate that an I/O throughput of 10 GB/s saturates the CPU
for many algorithms, even in a large parallel NUMA machine. 

FlashR simplifies the programming effort of writing parallel and out-of-core
implementations for large-scale machine learning. With FlashR, machine learning
researchers can prototype algorithms in a familiar programming environment,
while still getting efficient and scalable implementations.
We believe FlashR provides new opportunities for developing large-scale
machine learning algorithms.
